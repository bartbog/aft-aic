\paragraph{Lattices, Operators and Fixpoints}
%% \todo{RVWR 2: Could you provide an example of a 'grounded' point?
%% Could you provide examples for the various types of fixpoints (A-Kripke-Kleene fixpoint etc.)?}
%% \bart{What do we do with this one? If we want to give ``real'' examples, we should already introduce logic programming preliminaries in this section. 
%% Otherwise, we can only give abstract examples, which are quite hard to grasp.
%% Moving LP prelims here and giving some examples might work to me. Opinions? }
 \luis{I moved and reworded the last sentence of the section to the beginning.}
 \bart{That does not really solve his problem, of course, it just argues that we won't do it. 
 In itself, moving LP has the advantage that a reader sees AFT applies in a domain he might be familiar with, wether he comes from the KR or from the DB community, he is probably acquainted with some LP preliminaries. So, if we do it here, AFT might be easier to grasp?}

In this section we summarize the ideas, definitions and main results from approximation fixpoint theory (AFT) that we use in the remainder of the paper.
Since we only use these concepts in the context of AICs, we defer the examples of the constructions we introduce to the next section.
%When we introduce semantics for AIC based on AFT in the next section, we provide examples of the different constructions considered here. 

A \emph{partially ordered set (poset)} $\langle L,\leq\rangle$ is a set $L$ equipped with a partial order $\leq$, i.e., a reflexive, antisymmetric, transitive relation. 
As usual, we write $x<y$ as abbreviation for $x\leq y \land x\neq y$.
If $S$ is a subset of $L$, then $x$ is an \emph{upper bound} (a \emph{lower bound}) of $S$ if for every $s\in S$, it holds that $s\leq x$ ($x\leq s$, respectively).
An element $x$ is a \emph{least upper bound} (a \emph{greatest lower bound}) of $S$ if it is an upper bound that is smaller than every other upper bound (a lower bound that is greater than every other lower bound, respectively).
If $S$ has a least upper bound (a greatest lower bound) we denote it $\lub(S)$ ($\glb(S)$, respectively).
%BART: is there any reason to write resp. instead of respectively? I don't like abbreviations in journal verisons :p 
%Also: apparently always needs to be used postfix https://en.wiktionary.org/wiki/resp.
As is custom, we sometimes call a greatest lower bound a \emph{meet}, and a least upper bound a \emph{join} and use the related notations $\bigand S = \glb(S)$, $x\land y=\glb(\{x,y\})$, $\bigor S = \lub(S)$ and $x\lor y=\lub(\{x,y\})$.
% The relation $\leq$ is a \emph{total order} if for every $x,y\in L$, $x\leq y$ or $y\leq x$. 
% A \emph{chain} is a subset $S$ of $L$ such that $\leq$ is a total order in $S$. We call $\langle L,\leq\rangle$ \emph{chain complete} if each of its chains has a least upper bound and a greatest lower bound. 
%  We call $\langle L,\leq\rangle$ a \emph{bounded lattice}  if every finite subset of $L$ has a least upper bound and a greatest lower bound. 
 We call $\langle L,\leq\rangle$ a \emph{complete lattice}  if every subset of $L$ has a least upper bound and a greatest lower bound. 
%  A complete lattice is always chain complete.
% A chain complete poset has a least element $\bot$. 
A complete lattice has both a least element $\bot = \bigand L$ and a greatest element $\top = \bigor L$. 

A lattice $L$ is \emph{distributive} if $\land$ and $\lor$ distribute over each other, i.e., if $x\land(y\lor z)=(x\land y)\lor(x\land z)$ and $x\lor(y\land z)=(x\lor y)\land(x\lor z)$ for all $x,y,z\in L$.
A bounded lattice $L$ is \emph{complemented} if every element $x\in L$ has a complement: an element $\lnot x \in L$
satisfying $x\land \lnot x = \bot $ and $x\lor \lnot x = \top$. A \emph{Boolean lattice} is a distributive
complemented lattice.

Since we apply our results to (finite) databases, for the sake of simplicity we assume $L$ to be \emph{finite} in this text. All presented results easily generalize to the infinite setting as well. 

% A lattice is \emph{distributive} if $\land$ and $\lor$ distribute over each other. A bounded lattice is \emph{complemented} if every element $x\in L$ has a \emph{complement}: an element $\lnot x \in L$ satisfying $x\land \lnot x  = \bot$ and $x\lor \lnot x =\top$.
% A \emph{Boolean lattice} is a distributive complemented lattice. 
% In a Boolean lattice, for every $x,y \in L$, it holds that $x=(x\land y) \lor (x\land \lnot y)$.


An operator $O:L\to L$ is \emph{monotone} if $x\leq y$ implies that $O(x)\leq O(y)$.
%and \emph{anti-monotone} if $x\leq y$ implies that $O(y)\leq O(x)$. 
An element $x\in L$ is 
%a \emph{prefixpoint}, 
a \emph{fixpoint}
%, a \emph{postfixpoint} 
of $O$ if
%$O(x)\leq x$, respectively 
$O(x)=x$.
%, $x\leq O(x)$. 
Every monotone operator $O$ in a %chain complete poset 
complete lattice has a least fixpoint, denoted $\lfp(O)$, which is 
% also $O$'s least prefixpoint and 
the limit (the least upper bound) of the increasing sequence $(x_i)_{i \in \nat}$ defined by $x_0=\bot$ and $x_{i+1} = O(x_i)$. 
% \begin{itemize}
% 	\item $x_0=\bot$,
% 	\item $x_{i+1}=O(x_i)$, for successor ordinals $i+1$,
% 	\item $x_\lambda=\lub(\{x_i\mid i<\lambda\})$, for limit ordinals $\lambda$.
% \end{itemize}


\mycitet{GroundedFixpoints} called a point  $x\in L$ \emph{grounded} for $O$ if, for each $v\in L$ such that $O(v\land x)\leq v$, it holds that $x\leq v$. They called a point $x\in L$ \emph{strictly grounded} if there does not exist a $y$ such that $y<x$, and $O(y)\land x \leq y$. 
%   They called a point $x\in L$ \emph{strictly grounded} for $O$ if there is no $v\in L$ such that $v<x$ and $O(v)\land x \leq v$.
They explained the intuition underlying these concepts under the assumption that the elements of $L$ are sets of ``facts'' of some kind and the $\leq$ relation is the subset relation between such sets:
in this case, a point $x$ is grounded if it contains only facts that are sanctioned by the operator $O$, 
in the sense that if we remove them from $x$, then the operator will add at least one of them again. 
\citet{phd/Bogaerts15} showed that for Boolean lattices the notions of groundedness and strict groundedness coincide. In this paper, all lattices of our application are Boolean, hence we use both notions interchangeably.
% They also called a point $x\in L$ \emph{
% They showed that for Boolean lattices, the notions of \emph{groundedness} and \emph{strict groundedness} coincide. 
% In this text, we are only concerned with Boolean lattices and hence freely switch between the two definitions. 

\paragraph{Approximation Fixpoint Theory}


Given a lattice $L$, approximation fixpoint theory (AFT) \cite{DeneckerMT00} uses the bilattice 
$L^2$.  We define two \emph{projection} functions for pairs as usual:
$(x,y)_1=x$ and $(x,y)_2=y$.  Pairs $(x,y)\in L^2$ are used to
approximate elements in the interval $[x,y] = \{z\mid x\leq
z\wedge z\leq y\}$. We call $(x,y)\in L^2$ \emph{consistent} if $x\leq 
y$, that is, if $[x,y]$ is non-empty, and use $L^c$ to denote the set
of consistent elements. Elements $(x,x) \in L^c$ are called
\emph{exact}; they constitute the embedding of $L$ in $L^2$.  We sometimes abuse notation and use the tuple $(x,y)$
and the interval $[x,y]$ interchangeably.  The \emph{precision
  ordering} on $L^2$ is defined as $(x,y) \leqp (u,v)$ if $x\leq u$
and $v\leq y$. In case $(u,v)$ is consistent, this means that $(x,y)$
approximates all elements approximated by $(u,v)$, or in other words
that $[u,v]\subseteq [x,y]$.  If $L$ is a complete lattice, then
$\langle L^2,\leqp\rangle$ is also a complete lattice.
  
% \nomenclature[leqp]{$\leqp$}{The precision ordering on $L^2$}
% \nomenclature[A]{$A$}{An approximator of $O$}


AFT studies fixpoints of lattice operators $O:L\ra L$ through operators approximating $O$.
 An operator $A: L^2\to L^2$  is an \emph{approximator} of $O$ if it is \leqp-monotone,  and has the property that $A(x,x) = (O(x),O(x))$ for all $x$. %[x',y']$, where $(x',y')=A{x,x}$.
Approximators are internal in $L^c$ (i.e., map $L^c$ into $L^c$).
As usual, we often restrict our attention to \emph{symmetric} approximators: approximators $A$ such that, for all $x$ and $y$, $A(x,y)_1 = A(y,x)_2$.
%\cite{lpnmr/DeneckerV07}.
\citet{DeneckerMT04} showed that the consistent fixpoints of interest (supported, stable, well-founded) are uniquely determined by an approximator's restriction to $L^c$, hence, sometimes we only define approximators on $L^c$. 

AFT studies fixpoints of $O$ using fixpoints of $A$. 
 \begin{itemize}
  \item The \emph{$A$-Kripke-Kleene fixpoint} is the $\leqp$-least fixpoint of $A$, and it approximates all fixpoints of $O$. 
\item A \emph{partial $A$-stable fixpoint} is a pair  $(x,y)$ such that $x=\lfp(A(\cdot,y)_1)$ and $y=\lfp(A(x,\cdot)_2)$, where $A(\cdot,y)_1$ denotes the operator $L\to L:x\mapsto A(x,y)_1$ and analogously for $A(x,\cdot)_2$. 
\item The \emph{$A$-well-founded fixpoint} is the least precise (i.e., the $\leqp$-minimal) partial $A$-stable fixpoint.
% \luis{maybe define ``least precise''? I'm guessing minimal wrt $\leqp$?}
\item  An \emph{$A$-stable fixpoint} of $O$ is a fixpoint $x$ of $O$ such that $(x,x)$ is a partial $A$-stable fixpoint. This is equivalent to the condition that $x=\lfp(A(\cdot, x)_1)$.
\item A \emph{partial $A$-grounded fixpoint} is a consistent pair $(x,y)$ such that for each $v\in L$, whenever $A(x\land v, y\land v)_2 \leq v$, also $y\leq v$. 
 \end{itemize}
 
%  Several relationships exist between these fixpoints. 
 All (partial $A$-)stable fixpoints are (partial $A$-)grounded fixpoints and the $A$-well-founded fixpoint is the least precise partial $A$-grounded fixpoint \mycite{PartialGroundedFixpoints}. 
The $A$-Kripke-Kleene fixpoint of $O$ can be constructed as the limit of any monotone induction of $A$. 
For the $A$-well-founded fixpoint, a similar constructive characterization has been worked out by \citet{lpnmr/DeneckerV07}:

\begin{definition}\label{002:def:refinement}
An \emph{$A$-refinement} of $(x,y)$ is a pair $(x',y')\in L^2$ satisfying one of the following two conditions:
\begin{enumerate}
	\item $(x,y)\leqp(x',y')\leqp A(x,y)$, or \label{first}
	\item $x'=x$ and  $A(x,y')_2\leq y'\leq y$. \label{second}
\end{enumerate}
An $A$-refinement is \emph{strict} if $(x,y)\neq (x',y')$.
\end{definition}

We call the first type (\textbf{\ref{first}}) of refinements \emph{application refinements} and the second type (\textbf{\ref{second}}) \emph{unfoundedness refinements}. If $(x',y')$ is an $A$-refinement of $(x,y)$ and $A$ is clear from the context, we often write $(x,y)\to(x',y')$.
%
% \nomenclature[alp]{$\alpha,\beta$}{Ordinal numbers}
% \nomenclature[lam]{$\lambda$}{A limit ordinal}

 \begin{definition}
 A \emph{well-founded induction} of $A$  is a sequence 
$(x_i,y_i)_{i\leq n}$
with $n\in\nat$ such that 
\begin{itemize}
	\item $(x_0,y_0) = (\bot,\top)$;
	\item $(x_{i+1},y_{i+1})$ is an A-refinement of $(x_{i},y_{i})$, for  all $i<n$.
% 	\item $(x_{\lambda},y_{\lambda})= \lub_{\leqp} \{(x_i,y_i)\mid i<\lambda\}$
% 	      for each limit ordinal $\lambda\leq\beta$.
\end{itemize}
A well-founded induction is \emph{terminal} if its limit $(x_n,y_n)$ has no strict $A$-refinements.
\end{definition}
A well-founded induction is an algebraical generalization of the well-founded model construction defined by \citet{GelderRS91}. 
The first type of refinement corresponds to making a partial structure more precise by applying Fitting's immediate consequence operator; the second type of refinement corresponds to making a structure more precise by eliminating an unfounded set. 
For a given approximator $A$, there are many different terminal well-founded inductions of $A$.
\citet{lpnmr/DeneckerV07}  showed that they all have the same limit, which equals the $A$-well-founded fixpoint of $O$. Furthermore, if $A$ is symmetric, then the $A$-well-founded fixpoint of $O$ (in fact, every tuple in a well-founded induction of $A$) is consistent.
% Well-founded inductions that only use the first sort of refinement converge to the $A$-Kripke-Kleene fixpoint. 

% \todo{Include section on ultimate approximators?}
% The precision order can be pointwise extended to the family of approximators of $O$. It then follows that more precise approximators have a more precise well-founded fixpoint and that they have more stable fixpoints. 
% \cite{DeneckerMT04} showed that there exists a most precise approximator, $U_O$, called the ultimate approximator of $O$. 
% This operator is defined by \[U_O: L^c\to L^c: (x,y)\mapsto (\bigand O([x,y]), \bigor O([x,y])).\]
% % \nomenclature[UO]{$U_O$}{The ultimate approximator of $O$}
% Here, we used the notation $O(X) = \{O(x)\mid x\in X\}$ for a set $X\subseteq L$.
%  It then follows that for every
% approximator $A$, all  $A$-stable fixpoints are $U_O$-stable fixpoints, and  the $U_O$-well-founded fixpoint is always more precise than the $A$-well-founded fixpoint.  
% We refer to $U_O$-stable fixpoints as \emph{ultimate stable fixpoints} of $O$ and to the $U_O$-well-founded fixpoint as the \emph{ultimate well-founded fixpoint} of $O$.
% Semantics defined using the ultimate approximator have as advantage that they only depend on $O$ since the approximator can be derived from $O$.
% % In this paper, we will focus only on ultimate approximations. More specifically, we will show that for auto-epistemic logic, 
% % whose semantics is defined in terms of Approximation Fixpoint Theory, the ultimate approximator is not precise enough!
% % We will develop an alternative fixpoint theory that is more precise than the current one, and show that with this theory, we obtain several desirable properties in logics with a fixpoint semantics.

%% When we introduce semantics for AIC based on AFT in the next section, we provide examples of the different constructions considered here. 

%%% Local Variables:
%%% mode: latex
%%% TeX-master: "../AFT-semantics-AIC.tex"
%%% End:
