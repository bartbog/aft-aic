\begin{abstract}
Active integrity constraints (AICs) constitute a formalism to associate with a database not just the constraints it should adhere to, but also how to fix the database in case one or more of these constraints are violated. 
The intuitions regarding which repairs are ``good'' given such a description are closely related to intuitions that live in various areas of non-monotonic reasoning, such as logic programming and autoepistemic logic. 

In this paper, we apply \emph{approximation fixpoint theory}, an abstract, algebraic framework designed to unify semantics of non-monotonic logics, to the field of AICs.
This results in a new family of semantics for AICs. 
We study properties of our new semantics and relationships to existing semantics.
In particular, we argue that the AFT-well-founded semantics has some very desirable properties. 
\end{abstract}

\begin{keyword}
  active integrity constraints; approximation fixpoint theory
\end{keyword}
