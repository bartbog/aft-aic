In this paper, we defined an approximator in the domain of active integrity constraints. 
The result is a family of semantics for AICs based on existing intuitions in various domains of non-monotonic reasoning. 
We studied properties of our induced semantics. 
In particular the AFT-well-founded semantics possesses desirable properties: it approximates all repairs of various families (stable, justified, grounded) and hence can be used for approximate skeptical query-answering with respect to any of these semantics. 
Furthermore, the AFT-well-founded repair can be computed in time polynomial in the size of the database. 

Our study is far from finished. 
In the context of approximation fixpoint theory, \emph{ultimate approximators} have been studied by \citet{DeneckerMT04}. 
They showed that with each two-valued operator, we can associate a canonical approximator. 
The ultimate approximator induces another family of semantics for AICs. 
In other domains, e.g., in logic programming, semantics based on ultimate approximators have some very desirable properties, but in general come at the cost of a higher computational complexity than their ``standard'' variants. 
It remains to be researched if the same holds in the context of AICs. 
In this paper, we showed that the class of justified repairs is situated in between the classes of stable and of grounded repairs. 
It is known from AFT that the class of ultimate stable fixpoints also falls in between the classes of stable fixpoints (for any approximator) and grounded fixpoints. 
Hence, an interesting research question would be to verify if justified repairs coincide with ultimate stable fixpoints in this domain, and if not, how they relate. 
Another topic with potential for interesting future work are inconsistencies. Consider for instance the set of AICs $\{\lnot a \aicrule \add a, a \aicrule \remove a\}$; intuitively, we expect a semantic operator to derive an inconsistency from \emph{any} partial action set; in standard AFT this is not possible. However, extensions of AFT that accommodate this have been defined \cite{RR/BiJF14}; it would be interesting to see how AICs fit in this general theory. 
Another AFT-based topic of interest could be to study what \emph{safe inductions} \cite{ijcai/BogaertsVD17} yield in the context of AICs and whether they can fix problems with the well-founded semantics. 
One last topic on which more extensive research might be needed is the domain of revision programming \cite{tcs/MarekT98}. \citet{tplp/CaropreseT11} showed structural correspondences between semantics for AICs and semantics for revision programs. 
Our paper now paves the way to applying AFT to revision programming as well. 

% 
% 
% \todo{Say something about future work:
% 
% ultimate approximations. Currently, we found
% $stable \subseteq justified\subseteq grounded$ and all these inclusions can be strict. 
% It also holds that 
% $stable \subseteq ult-stable\subseteq grounded$ and all these inclusions can be strict. 
% So the question naturally arises: how do justified repairs compare to ultimate stable fixpoints. Also, the u-wff is more precise than the standard wff. So it gives us more information and can also be useful in practice. 
% }
% 
% 
% \todo{say somehting about the impact on ``revision programming''}

%%% Local Variables:
%%% mode: latex
%%% TeX-master: "../AFT-semantics-AIC.tex"
%%% End:
