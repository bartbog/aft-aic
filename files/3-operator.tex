In this section we show how a set of normal AICs induces an operator on a suitably defined lattice.

% \subsection{The opera	tor $\Op$}

Given a fixed database $\db$, we are interested in the sets of update actions $\UU$ such that:
\begin{enumerate}
\item \UU is consistent and 
\item each action in \UU modifies \db.
\end{enumerate}
If $a\in \atoms$ and $\db$ is a database, we define $\change a$ to be the update action $\add a$ if $a\not\in \db$ and $\remove a$ if $a \in \db$.
This establishes a bijection between the sets of update actions satisfying the above two conditions and subsets of $\atoms$, where a set of atoms $\overline\UU\subseteq\atoms$ is identified with the set 
\[\UU = \{\change a\mid a \in \overline \UU \}\]
of update actions.
When no ambiguity arises, we omit the bar, and simply write $\UU$ for the subset of \atoms as well.
Only in places where the distinction between the two is essential (in technical parts of proofs) do we explicitly write $\overline\UU$.

Following the principle of minimality of change~\cite{Winslett90,ai/EiterG92}, we also typically prefer smaller sets of updates over larger sets.
Therefore, we are interested in the lattice $\langle 2^{\atoms},\subseteq\rangle$, where smaller elements correspond to better repairs according to this principle.

The intuitive reading of an AIC $r$ naturally suggests an operator over this lattice, defined as ``if $\UU(\db)\models\body(r)$ holds, then add $\head(r)$ to $\UU$''.
However, this definition quickly leads to inconsistent sets of update actions upon iteration, which we want to avoid.
We therefore propose a slight variant of this intuition, using the following concept.

\begin{definition}
  Let $\UU_1$ and $\UU_2$ be consistent sets of update actions over a set of atoms $\atoms$.
  The set $\UU_1\uplus\UU_2$ is defined as
  \[\UU_1\uplus\UU_2 = (\UU_1\cup\{\alpha\in\UU_2\mid\alpha^D\not\in\UU_1\})\setminus\{\alpha\in\UU_1\mid\alpha^D\in\UU_2\}\,.\]
\end{definition}
This operation models sequential composition of repairs in the following sense: given a database $\db$, if every action in $\UU_1$ changes $\db$ and every action in $\UU_2$ changes $\UU_1(\db)$, then $(\UU_1\uplus\UU_2)(\db)=\UU_2(\UU_1(\db))$.
In this case, we also have the relation
\[\left(\UU_1\uplus \UU_2\right)(\db) = \left(\UU_1\cup \UU_2\right) \setminus \{\alpha \mid \alpha,\alpha^D\in \UU_1\cup\UU_2\}\,.\]
Furthermore, if $\UU_1$ and $\UU_2$ are both consistent, then so is $\UU_1\uplus\UU_2$.

\begin{definition}
  Let $\db$ be a database and $\eta$ be a set of AICs over $\db$.
  The operator $\Op^\db:2^\atoms\to 2^\atoms$ is defined as follows:
  \[
    \Op^\db(\UU) = \UU\uplus\{\head(r)\mid \UU(\db)\models\body(r)\}
  \]
\end{definition}
In other words, $\Op^\db(\UU)$ is obtained by updating $\UU$ with the heads of all AICs whose bodies are satisfied by $\UU(\db)$.
The resulting set may contain less update actions than the number of such rules, since different AICs may include the same action in their heads.

\luis{There was a comment about defining this with $\nup(r)$ instead. I'm not sure this is correct.}

\luis{add example here}
\begin{example}[Example \ref{ex:repairs} continued]
  Consider again the set of AICs $\eta$ from Example~\ref{ex:repairs}, where
  $\db=\{a,b\}$.
  Then $\Op(\emptyset)=\{\remove a, \remove b\}$. Indeed, the bodies of all rules are satisfied; hence all heads are ``added'' to $\emptyset$. 
\end{example}

The syntactic restrictions on AICs guarantee that the set $\Op^\db(\UU)$ is always consistent: if $\{+a,-a\}\subseteq\Op^\db(\UU)$, then there are rules $r_1$ and $r_2$ such that $\neg a\in\body(r_1)$ and $a\in\body(r_2)$ with $\UU(\db)\models\body(r_i)$ for $i=1,2$, which is impossible.
In the interest of legibility, we write simply $\Op$ instead of $\Op^\db$ whenever $\db$ is clear from the context.

The operator $\Op$ provides alternative characterizations of the notions of weak repair, repair, founded, well-founded and grounded sets of update actions.

\begin{lemma}
  \label{lem:weak-repair}
  Let $\db$ be a database, $\eta$ be a set of normal AICs over $\db$ and $\UU$ be a set of update actions.
  Then $\UU$ is a weak repair for $\fulldb$ if and only if $\UU$ is a fixpoint of $\Op$.
\end{lemma}
\begin{proof}
  If $\UU$ is a weak repair for $\fulldb$, then $\UU(\db)\not\models\body(r)$ for all $r\in\eta$, whence $\Op(\UU)=\UU$.
  If $\UU$ is not a weak repair for $\fulldb$, then $\UU(\db)\models\body(r)$ for some $r\in\eta$, and $\Op(\UU)$ differs from $\UU$ by (at least) $\head(r)$.
\end{proof}

\begin{lemma}
  \label{lem:repair}
  Let $\db$ be a database, $\eta$ be a set of normal AICs over $\db$ and $\UU$ be a set of update actions.
  Then $\UU$ is a repair for $\fulldb$ if and only if $\UU$ is a minimal fixpoint of $\Op$.
\end{lemma}
\begin{proof}
  Follows directly from Lemma~\ref{lem:weak-repair} and the definition of repair.
\end{proof}

\begin{lemma}
  \label{lem:founded-char}
  Let $\db$ be a database, $\eta$ be a set of normal AICs over $\db$ and $\UU$ be a consistent set of update actions.
  Then $\UU$ is founded with respect to $\fulldb$ if and only if, for all $\alpha\in\UU$, it is the case that $\alpha\in\Op(\UU\setminus\{\alpha\})$.
\end{lemma}
\begin{proof}
  An action $\alpha\in\UU$ is founded if and only if there is a rule $r\in\eta$ such that $\UU(\db)\models\body(r)\setminus\{\alpha^D\}$.
  This is equivalent to saying that $(\UU\setminus\{\alpha\})(\db)\models\body(r)$.
  But, by definition of $\Op$, we have $\alpha\in\Op(\UU\setminus\{\alpha\})$ if and only if there is a rule $r\in\eta$ such that $(\UU\setminus\{\alpha\})(\db)\models\body(r)$, which concludes the proof.

  Consistency of $\UU$ is needed for the direct implication, as $\alpha$ is only added to $\UU\setminus\{\alpha\}$ by $\Op$ if that set does not already contain $\alpha^D$.
\end{proof} 

This result also gives some intuition regarding why founded repairs allow for circular dependencies: the definition of founded repair only checks that each individual action is supported by the remaining ones, but it allows for dependency cycles.

\begin{lemma}
  \label{lem:wf}
  Let $\db$ be a database, $\eta$ be a set of normal AICs over $\db$ and $\UU$ be a weak repair for $\fulldb$.
  Then $\UU$ is well-founded if and only if there is an ordering $\alpha_1,\ldots,\alpha_n$ of the elements of $\UU$ such that $\alpha_i\in\Op(\{\alpha_1,\ldots,\alpha_{i-1}\})$ for each $i=1,\ldots,n$.
\end{lemma}
\begin{proof}
  If $\UU$ is well-founded, then the ordering is given by the sequence of actions introduced at each node in the path, in the well-founded repair tree for $\fulldb$, going from the root to the node with label $\UU$.
  Conversely, if $\UU$ can be obtained in the manner described, then it defines a valid path in that same tree ending at a leaf.
\end{proof}



\begin{lemma}
 Let $\db$ be a database and $\eta$ a set of AICs over \db. A set of update actions $\UU$ is a grounded repair of \fulldb if and only if $\UU$ is a grounded fixpoint of \Op. 
\end{lemma}
\begin{proof}
 First suppose \UU is a grounded repair of \fulldb. From Lemma \ref{lem:repair}, it follows that $\UU$ is a fixpoint of $\Op$. Suppose that $\UU$ is not a \emph{grounded} fixpoint of \Op. This means that there exists a set of update actions $V$ such that $\Op(\UU\cap V) \subseteq V$ and $\UU \not \subseteq V$. Taking $A=\UU\setminus V$, and $\UU'=\UU\setminus A$, we find that $\Op(\UU') \cap A = \emptyset$. This means (following the definition of \Op) that there is no rule whose body is satisfied  that derives an element of $A$ in $\UU'$. Since $\UU$ is a grounded repair, we arrive at a contradiction. 
 
 On the other hand, assume $\UU$ is a grounded fixpoint of \Op and let $A$ be any subset of $\UU$ such that no rule $r$ with $\head(r)\in A$ has $\body(r)$ satisfied in $\UU'=\UU\setminus A$. We should show that $A=\emptyset$. Now, from the definition of $\Op$, it follows that $\Op(\UU')\cap A=\emptyset$ and thus that $\Op(\UU') \cap \UU \subseteq \UU'$. Since groundedness and strict groundedness coincide, $\UU$ is strictly grounded and thus, $\UU'=\UU$ and $A=\emptyset$ indeed. 
\end{proof}







The correspondence between justified repairs and answer sets for particular logic programs~\cite{Caroprese2011} shows that justified repairs can also be characterized in a related manner.
However, since answer sets of a logic program are models of its Gelfond--Lifschitz transform, the corresponding characterization in terms would be as fixpoints of the corresponding operator for a similarly derived set of AICs, rather than of $\Op$.



\subsection{Grounded fixpoints of $\Op$}
\todo{lemma: agree}

\todo{sheds new light on Lemma \ref{lem:}}:
This result is not very surprising: justified weak repairs are answer sets of a particular logic program (Theorem 6 in~\cite{Caroprese2011}), and in turn answer sets of logic programs are grounded fixpoints of the consequence operator (see remark at the top of Section~5 in~\mycite{GroundedFixpoints}).
However, the translation defined in~\cite{Caroprese2011} is from logic programs to databases with AICs (rather than the other way around), so Lemma~\ref{lem:justified} is \emph{not} a direct consequence of those results.
Also, since grounded repairs can be built from the ground up, this result also establishes that justified repairs avoid circularity of support (a property that was claimed in~\cite{Caroprese2011}, but never formally discussed).


%%%%%%%%%%%%%%%%%%%%%%%%%

Founded, well-founded and justified repairs were all introduced with the purpose of characterizing a class of repairs whose actions are supported (there is a reason for having them in the set), and that support is not circular; in particular, these repairs should be constructible ``from the ground up'', which was the motivation for defining well-founded repairs.
However, all notions exhibit unsatisfactory examples: there exist founded repairs with circular support~\cite{Caroprese2011} and repairs with no circular support that are not justified~\cite{CEGN13}; well-founded repairs, on the other hand, are not stratifiable~\cite{lcf:14}, which impacts their computation in practice.

Following the intuition in~\mycite{GroundedFixpoints} that grounded fixpoints capture the idea of building fixpoints ``from the ground up'', we propose the following notion of $\Op$.

\begin{definition}
  Let $\db$ be a database, $\eta$ be a set of normal AICs over $\db$.
  A repair $\UU$ for $\fulldb$ is \emph{grounded} if $\UU$ is a grounded fixpoint of $\Op$.
\end{definition}

Since we are working within a powerset lattice, the notions of grounded and strictly grounded fixpoints coincide.
\luis{check that we talk about these}
As it turns out, the latter notion is most convenient for the proofs of our results.
We thus characterize grounded repairs as repairs $\UU$ such that: if $\UU'\subsetneq\UU$, then $\Op(\UU')\cap\UU\not\subseteq\UU'$.
Equivalently: if $\UU'\subsetneq\UU$, then $\Op(\UU')\cap(\UU\setminus\UU')\neq\emptyset$.

Since all grounded fixpoints are minimal, it makes no sense to define grounded weak repairs.
The notion of grounded fixpoint therefore intrinsically embodies the principle of minimality of change, unlike other kinds of weak repairs previously defined.
Furthermore, grounded repairs also embody the notion of ``support'' previously defined.

\begin{lemma}
  \label{lem:founded}
  Let $\db$ be a database, $\eta$ be a set of normal AICs over $\db$ and $\UU$ be a grounded repair for $\fulldb$.
  Then $\UU$ is both founded and well-founded.
\end{lemma}
\begin{proof}
  Assume that $\UU$ is a grounded repair for $\fulldb$.
  For each $\alpha\in\UU$, necessarily $\Op(\UU\setminus\{\alpha\})\cap\UU\subsetneq(\UU\setminus\{\alpha\})$, which implies that $\alpha\in\Op(\UU\setminus\{\alpha\})$.
  By Lemma~\ref{lem:founded-char}, this implies that $\UU$ is founded.

  Now construct the sequence for well-founded repairs always choosing $u_i\in\UU$ until this is no longer possible.
  Letting $\UU'$ be the last constructed set, by construction both $\UU'\subseteq\UU$ and $\Op(\UU')\cap\UU\subseteq\UU'$ (otherwise we could proceed).
  Since $\UU$ is grounded, it cannot be the case that $\UU'\subsetneq\UU$, so $\UU=\UU'$ and it is thus a well-founded repair.
\end{proof}

\luis{check example, add new one here}
However, the notion of grounded repair is strictly stronger than both of these: Example~\ref{ex:founded}, presented earlier, also shows that some forms of circular justifications are avoided by grounded repairs.
%% \begin{example}
%%   \label{ex:founded-gr}
%%   Recall the set of AICs from Example~\ref{ex:founded}, together with $\db=\{a,b\}$.
%%   The repair $\UU_1=\{-a,-b\}$ is a founded repair that is not grounded: $\UU'=\emptyset$ satisfies $\Op(\UU')\cap\UU_1=\{+c\}\cap\UU_1=\emptyset\subseteq\UU'$.
%%   The more natural repair $\UU_2=\{+c\}$ is also founded, and it is immediate to verify that it is also grounded.
%% \end{example}

Likewise, all well-founded repairs are grounded, but not conversely.
\luis{check example}
%% \begin{example}
%%   Consider again $\eta$ from Example~\ref{ex:well-founded}, with $\db=\emptyset$.
%%   As shown earlier, the two well-founded repairs for $\fulldb$ are $\UU_1=\{{+b},{+a}\}$ and $\UU_2=\{{+a},{+c}\}$.
%%   We already observed that $\UU_2$ is not founded, so it cannot be grounded; indeed, $\UU'=\{{+a}\}$ is a strict subset of $\UU_2$, and $\Op(\UU')\cap\UU_2=\{{+a},{+b}\}\cap\UU_2=\emptyset\subseteq\UU'$.
%% \end{example}

We now investigate the relation to justified repairs, and find that all justified repairs are grounded, but not conversely -- confirming our earlier claim that the notion of justified repair is too strong.

\begin{lemma}
  \label{lem:justified}
  Let $\db$ be a database, $\eta$ be a set of normal AICs over $\db$ and $\UU$ be a justified repair for $\fulldb$.
  Then $\UU$ is grounded.
\end{lemma}
\begin{proof}
  Let $\UU$ be a justified repair for $\fulldb$ and assume that $\UU'\subsetneq\UU$.
  Then $\UU'\cup\neff(\UU)$ is not closed under $\eta$, whence there is a rule $r\in\eta$ such that $\actof(\nup(r))\subseteq\UU'\cup\neff(\UU)$ and $\head(r)\not\in\UU'\cup\neff(\UU)$.

  Since $\UU'\subseteq\UU$, also $\actof(\nup(r))\subseteq\UU\cup\neff(\UU)$, whence $\head(r)\in\UU\cup\neff(\UU)$ as $\UU$ is closed under $\eta$.
  But $\head(r)\not\in\UU'\cup\neff(\UU)$, so $\head(r)\in\UU\setminus\UU'$.

  Then $\UU'\models\body(r)$: on the one hand, $\actof(\nup(r))\subseteq\UU'\cup\neff(\UU)$ implies that $\UU'(\db)\models\nup(r)$, as $\neff(\UU)\subseteq\neff(\UU')$; on the other hand, from $\head(r)\in\UU$ we know that $\litof(\head(r))^D\in\db$ (all actions in $\UU$ change $\db$), whence $\UU'(\db)\models\litof(\head(r)^D)$ since $\head(r)\not\in\UU'$.
  As $r$ is normal, there are no more literals in $\body(r)$, so $\UU'(\db)\models\body(r)$ and therefore $\head(r)\in\Op(\UU')$.

  We thus conclude that $\Op(\UU')\cap\UU\not\subseteq\UU'$.
  By arbitrariness of $\UU'$, it follows that $\UU$ is grounded.
\end{proof}
This proof may look suspicious, as it does not use the hypothesis that $\UU$ is a repair.
Indeed, Theorem~4 of~\cite{Caroprese2011} shows that all justified weak repairs are minimal when $\eta$ contains only normal AICs.

This result is not very surprising: justified weak repairs are answer sets of a particular logic program (Theorem 6 in~\cite{Caroprese2011}), and in turn answer sets of logic programs are grounded fixpoints of the consequence operator (see remark at the top of Section~5 in~\mycite{GroundedFixpoints}).
However, the translation defined in~\cite{Caroprese2011} is from logic programs to databases with AICs (rather than the other way around), so Lemma~\ref{lem:justified} is \emph{not} a direct consequence of those results.
Also, since grounded repairs can be built from the ground up, this result also establishes that justified repairs avoid circularity of support (a property that was claimed in~\cite{Caroprese2011}, but never formally discussed).

Recall Example~\ref{ex:justified}, which was used in~\cite{CEGN13} to point out that justified repairs sometimes eliminate ``natural'' repairs.
This example also shows that the notion of justified repair is also stricter than that of grounded repair.
\luis{check example}
%% \begin{example}
%%   Recall the set $\eta$ from Example~\ref{ex:justified}, with $\db=\{a,b\}$.
%%   Although the repair $\UU=\{-a,-b\}$ for $\fulldb$ is not justified, it is grounded: if ${-a}\in\UU'\subsetneq\UU$, then $\Op(\UU')\cap\UU$ contains ${-b}\in\UU\setminus\UU'$, else $\Op(\UU')\cap\UU$ contains ${-a}\in\UU\setminus\UU'$.
%% \end{example}
%% As discussed earlier, in this case the first rule clearly motivates the action $-a$, and the last rule then requires $-b$.
%% This is in contrast to Example~\ref{ex:founded}, where there was no clear reason to include either $-a$ or $-b$ in a repair.
%% So grounded repairs avoid this type of unreasonable circularities, without being as restrictive as justified repairs.

We thus have that grounded repairs are always founded and well-founded; the next example shows that they do not correspond to the intersection of those classes.

\begin{example}
  Consider the following set of AICs $\eta$.
  \begin{align*}
    \neg a,\neg b&\supset{+a} \qquad (r_{13}) \qquad &
    a,\neg b&\supset{+b} \qquad (r_{14}) \qquad &
    \neg a,b&\supset{-b} \qquad (r_{15}) \\
    a,b,\neg c&\supset{+c} \qquad (r_{16}) \qquad &
    a,\neg b,c&\supset{+b} \qquad (r_{17}) \qquad &
    \neg a,b,c&\supset{+a}\qquad (r_{18})
  \end{align*}
  Let $\db=\emptyset$.
  Then $\UU=\{{+a},{+b},{+c}\}$ is a repair for $\fulldb$: the first row of constraints requires ${+a}$ and ${+b}$ to be included in the database, and the second row states that no $2$-element subset of $\UU$ can be a repair.
  Furthermore, $\UU$ is founded (the rules in the second row ensure that) and well-founded (starting with $\emptyset$, we are forced to apply rules $r_{13}$, $r_{14}$ and $r_{16}$, in that order).

  However, $\UU$ is not strictly grounded for $\Op$: if $\UU'=\{{+b}\}$, then $\UU'\subsetneq\UU$, but $\Op(\UU')\cap\UU=\emptyset\cap\UU=\emptyset\subseteq\UU'$.
\end{example}
In this situation, $\UU$ actually seems reasonable; however, observe that the support for its actions \emph{is} circular: it is the three rules in the second row that make $\UU$ founded, and none of them is applicable to $\db$.
Also note that $\UU'(\db)$ is a database for which the given set $\eta$ behaves very awkwardly: the only applicable AIC tells us to remove $b$, but the only possible repair is actually $\{{+a},{+c}\}$.

We do not feel that this example weakens the case for studying ground repairs, though: the consensual approach to different notions of repair is that they express \emph{preferences}.
In this case, where $\fulldb$ admits no grounded repair, it is sensible to allow a repair in a larger class -- and a repair that is both founded and well-founded is a good candidate.
The discussion in Section~8 of~\cite{Caroprese2011} already proposes such a ``methodology'': choose a repair from the most restrictive category (justified, founded, or any).
We advocate a similar approach, but dropping justified repairs in favor of grounded repairs, and preferring well-founded to founded repairs.

We summarize the relations between the different types of repairs in the following diagram.
%% \[\xymatrix@R-1em@C+2em{%
%%   \mathcal F \ar@{-}[dd]|{\neq} \ar@{-}[dr]|{\subsetneq} \\
%%   & \mathcal G \ar@{-}[r]|{\subsetneq} & \mathcal J \\
%%   \mathcal{WF} \ar@{-}[ur]|{\subsetneq}
%% }
%% \]
\luis{referee suggested Venn chart, I have one somewhere}

We conclude this section with a note on complexity.

\begin{theorem}
  \label{thm:grounded-complexity}
  Let $\db$ be a database and $\eta$ be a set of normal AICs over $\db$.
  The problem of deciding whether there exist grounded repairs for $\fulldb$ is $\Sigma^P_2$-complete.
\end{theorem}
\begin{proof}
  For membership, we need to show that we can decide the problem with a non-deterministic Turing machine with an NP oracle.
  Given a set of update actions $\UU$, checking that it is a fixpoint of $\Op$ can be done in polynomial time on the size of $\db$ and $\eta$; the NP-oracle can then answer whether there exists $\UU'\subsetneq\UU$ with $\Op(\UU')\cap\UU\subseteq\UU'$, thereby establishing whether $\UU$ is grounded.

  For hardness, we invoke the (polynomial time) translation $\mathit{aic}$ from logic programs to sets of AICs over the empty database given Section~7 of~\cite{Caroprese2011}.
  Given a logic program $\mathcal P$, deciding whether $\langle\emptyset,\mathit{aic}(\mathcal P)\rangle$ has a grounded repair is equivalent to deciding whether $\mathcal P$ has a grounded model, which is $\Sigma^P_2$-complete by Theorem~5.7 of~\mycite{GroundedFixpoints}.
\end{proof}

This result still holds if we allow a truly first-order syntax for AICs, where the atoms can include variables that are implictly universally quantified.

%%% Local Variables:
%%% mode: latex
%%% TeX-master: "../AFT-semantics-AIC.tex"
%%% End:
