\luis{need to rewrite this whole section -- include the relevant stuff from the ICLP paper}

Following \citet{iclp/Cruz-Filipe16}, we define an operation $\biguplus$ on consistent sets of update actions: 
\[
\UU_1 \biguplus \UU_2 = (\UU_1 \cup \{\alpha \in \UU_2 \mid \alpha^D\not \in \UU_1 \}) \setminus \{\alpha\in \UU_1 \shortpaper{{\mid}}\mid \alpha^D\in \UU_2\}
\]
This operation has the property that: if every action in $\UU_1$ changes \db and every action in $\UU_2$ changes $\UU_1(\db)$, then 
\[\left(\UU_1\biguplus \UU_2\right)(\db) = \UU_2(\UU_1(\db))
= \UU_1\cup \UU_2 \setminus \{\alpha \mid \alpha,\alpha^D\in \UU_1\cup\UU_2\}.\]

\paragraph{A Semantic Operator for AICs}
Recall that in this paper, we only consider normal AICs.
% , i.e., expressions $r$ of the form 
% \[ l_1\land \ldots \land l_n \aicrule \alpha\]
% where $\alpha = \actof(l_1)^D$ (after renumbering the $l_i$).
% For such a rule, we define $\nup(r) = \l_2\land \dots \land l_n$. The intuition is that $\nup(r)$ is the non-updatable part of the body of $r$. 
% As soon as $\nup(r)$ holds, so should $l_1^D$. Indeed, if $l_1$ holds, then $\alpha$ dictates to change its value; if $l_1$ does not hold, the integrity constraint is satisfied. 

Given a fixed database $\db$,
the sets of update actions $\UU$ that are of interest to us are those such that 
\begin{inparaenum}
\item \UU is consistent and 
\item each action in \UU modifies \db.
\end{inparaenum}
As argued by  \citet{iclp/Cruz-Filipe16}, the set of such sets is isomorphic to $2^{\atoms}$. Hence, from now on, we identify such a set with a subset of $\atoms$ (the atoms whose value is changed by \UU). 
If $a\in \atoms$ and $\db$ is a database, we define $\change a$ to be the update action $\add a$ if $a\not\in \db$ and $\remove a$ if $a \in \db$. 
Thus, in the above identification, a set  of atoms $\overline\UU\subseteq \atoms$ is identified with the set 
\[\UU = \{\change a\mid a \in \overline \UU \}\]
of update actions.
Usually, we omit the bar, and simply write $\UU$ for the subset of \atoms as well.
Only in places where the distinction between the two is essential (in technical parts of proofs) do we explicitly write $\overline \UU$.

Following the principle of minimality of change \cite{Winslett90,ai/EiterG92}, we prefer smaller sets of updates. % over lager sets.
The lattice we are interested in is thus $\langle 2^{\atoms}, \subseteq\rangle$, where  smaller elements correspond to ``better'' repairs. 


Now consider a set of (normal) AICs \aics. 
\citet{iclp/Cruz-Filipe16} associated with such a set a semantic operator
% \footnote{Technically, his definition slightly differs from ours in the sense that $\nup$ is replaced by \body in his definition. However, it is not hard to see that the definitions are equivalent. The current definition satisifies our purpose better.} 
%\begin{align*}\Op: &2^{\atoms}\to  2^{\atoms}: \UU \mapsto  \\ &\UU \biguplus \{\head(r) \mid r\in\aics \land  \UU(\db)\models \body(r)\}.\end{align*}
$\Op: 2^{\atoms}\to  2^{\atoms}$ such that
\[\Op(\UU) = \UU \biguplus \{\head(r) \mid r\in\aics \land  \UU(\db)\models \body(r)\}.\]

Cruz-Filipe argued that a semantics for AICs based on grounded fixpoints of \Op coincides with the intuitions on a large set of  examples and that it solves problems with several previously existing semantics. 


%%% Local Variables:
%%% mode: latex
%%% TeX-master: "../AFT-semantics-AIC.tex"
%%% End:
