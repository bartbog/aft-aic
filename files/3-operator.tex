In this section we show how a set of normal AICs induces an operator on a suitably defined lattice.

% \subsection{The opera	tor $\Op$}

Given a fixed database $\db$, we are interested in the sets of update actions $\UU$ such that:
\begin{enumerate}
\item \UU is consistent and 
\item each action in \UU modifies \db.
\end{enumerate}
Note that the second condition here implies the first since it is not possible that both $\add a$ and $\remove a$ modify \db.
For each atom $a\in\atoms$, we define 
 \begin{align*}
 \change a &= \left\{\begin{array}{ll}
                      \add a & \text{if } a \not \in \db\\
                      \remove a & \text {otherwise.}
                     \end{array}\right.
\end{align*}
Let us furthermore denote the set of update actions that modify \db by \acts. 
With this notation, $\acts = \{\change a \mid a \in \atoms\}$ and the sets of update actions we are interested in are elements of $2^\acts$. 
Note that $\change$ and \acts are defined solely based on the initial database \db and are not dependent of, for instance, a given repair.


% We denote the set of all such sets of update actions by \cua. 
% It is easy to see that an element of \cua is uniquely identified by the set of atoms of $\db$ it changes, i.e., that elements of $\cua$ correspond one-to-one to elements of $2^{\atoms}$. 
% Formally, we define the following two mappings that establish this correspondence
% \begin{align*}
% \ats&: \cua \to 2^{\atoms}: \UU \mapsto \{a\in \atoms \mid \add a \in \UU \text{ or } \remove a \in \UU\}\\
%  \acts &: 2^{\atoms}\to \cua: \UU \mapsto \{\change a \mid a \in \UU \},
% \end{align*}
% where 
% \begin{align*}
%  \change a &= \left\{\begin{array}{ll}
%                       \add a & \text{if } a \not \in \db\\
%                       \remove a & \text {otherwise.}
%                      \end{array}\right.
% \end{align*}




% % $\litof$ $\actof$
% % $\atomof$ $\action$
% For mathematical simplicity, it is often easier to work with $2^{\atoms}$ instead of working directly with sets of repairs. Hence, our operators will be defined on this set. However, we would like to stress that such a set should always be thought of as a set of update actions. 


% If $a\in \atoms$ and $\db$ is a database, we define 
% This establishes a bijection between the sets of update actions satisfying the above two conditions and subsets of $\atoms$, where a set of atoms $\overline\UU\subseteq\atoms$ is identified with the set 
% \[\UU = \{\change a\mid a \in \overline \UU \}\]
% of update actions.
% When no ambiguity arises, we omit the bar, and simply write $\UU$ for the subset of \atoms as well.
% Only in places where the distinction between the two is essential (in technical parts of proofs) do we explicitly write $\overline\UU$.
% \bart{There are several problems with this identification:
% 
% RVWR 1:  $\bar U$ is identified with U.  But U is defined on line 16 in terms of $\bar U$. Maybe define $\bar U$ in terms of U : $\bar U$ = {a | ch a in U}
% 
% TODO: emphasize taht ch a does not depend on U
% 
% }

Following the principle of minimality of change~\cite{Winslett90,ai/EiterG92}, we also typically prefer smaller sets of updates over larger sets.
Therefore, we are interested in the lattice $\langle 2^{\acts},\subseteq\rangle$, where smaller elements correspond to better repairs according to this principle.


The intuitive reading of an AIC $r$ naturally suggests an operator over this lattice, defined as ``if $\UU(\db)\models\body(r)$, then add $\head(r)$ to $\UU$'' to obtain a new element of $2^{\acts}$. However, this naive definition does not lead to an operator that is internal in $2^{\acts}$, as illustrated for instance in the following example.
\begin{example}
 Consider 
 \begin{align*}
  \aics = \{ \lnot a \aicrule \add a\} 
 \end{align*}
 and $\db=\{a\}$. 
 In this case, $\acts = \{\remove a\}$ and $2^{\acts} = \{ \emptyset, \{\remove a\}\}$.
 Now, taking $\UU = \{\remove a\}$, the body of the only rule in $\aics$ is satisfied.
 Naively adding its head to $\UU$ results in the set $\{\add a,\remove a\}$, which is not an element of $2^\acts$. 
 We expect a semantic operator to map $\UU$ to $\emptyset$ since the rule in \aics indicates that the problems with $\UU(\db)$ can be solved by adding $a$, i.e., by not changing $\db$ at all. 
\end{example}
This kind of problems is inherent to the fact that AICs can have rules with dual heads and does not occur in other formalisms where AFT is applied, such as, e.g., logic programming\footnote{Notice that for instance in default logic, rules with complementary literals in their head are allowed. However, such literals are treated as two independent statements and can be derived separately. In the current setting, an update action $\add p$ can undo the effect of a previously derived action $\remove p$.}. 
Intuitively, the operator $\Op$ we wish to define should satisfy the following properties:
\begin{itemize}
 \item \emph{(inertia)} Only change something in the input if there is a rule that warrants this change. 
 This requirement consists itself of two parts: 
 \begin{itemize}
 \item Do not add anything to $\UU$ unless there is a reason for it, i.e., \[\Op(\UU) \subseteq \UU \cup \{\head(r)\mid r\in \aics \land \UU(\db)\models\body(r)\}.\] 
 \item Do not remove anything from $\UU$ unless there is a reason for it, i.e.,  \[\UU \setminus \{\head(r)^D\mid r\in \aics \land \UU(\db)\models\body(r)\} \subseteq \Op(\UU).\] \end{itemize}
 \item \emph{(cancellation)} If there is an action in \UU that is ``canceled out'' by some rule in \aics, then neither the action nor its dual are in the result (i.e., \db remains unchanged with respect to this action). Formally, if $\alpha\in \UU$ and $\alpha^D \in \{\head(r)\mid r\in \aics \land \UU(\db)\models\body(r)\}$, then $\alpha\not \in \Op(\UU), \alpha^D\not \in \Op(\UU)$. 
 \item \emph{(completeness)} If there is an applicable rule whose body is satisfied, and whose head does not contradict $\UU$, then the head is derived. Formally, 
 \[ \{\head(r)\mid r\in \aics \land \UU(\db)\models\body(r) \land \head(r)^D \not \in \UU \}\subseteq \Op(\UU).\]  %from which an action can be derived that does not contradict $\UU$, then 
\end{itemize}
It turns out that these three properties uniquely define an operator on $2^{\acts}$. 
In order to give a constructive characterization of this operator, we introduce the following concept. 

% However, this definition quickly leads to inconsistent sets of update actions upon iteration, which we want to avoid.
% We therefore propose a slight variant of this intuition, using the following concept that consists of taking the union and restoring consistency afterwards.

\begin{definition}
  Let $\UU_1$ and $\UU_2$ be sets of update actions over a set of atoms $\atoms$.
  The set $\UU_1\uplus\UU_2$ is defined as
  \[\left(\UU_1\uplus \UU_2\right)(\db) = \left(\UU_1\cup \UU_2\right) \setminus \{\alpha \mid \alpha,\alpha^D\in \UU_1\cup\UU_2\}\,.\]
  %\[\UU_1\uplus\UU_2 = (\UU_1\cup\{\alpha\in\UU_2\mid\alpha^D\not\in\UU_1\})\setminus\{\alpha\in\UU_1\mid\alpha^D\in\UU_2\}\,.\]
\end{definition}
This operation models sequential composition of repairs in the following sense: given a database $\db$, if every action in $\UU_1$ changes $\db$ and every action in $\UU_2$ changes $\UU_1(\db)$, then $(\UU_1\uplus\UU_2)(\db)=\UU_2(\UU_1(\db))$.

Observe that, while a set of AICs may include both a rule with head $+a$ and another with head $-a$, these rules are not simultaneously applicable, as the conjunction of their bodies is always unsatisfiable.
However, it may be the case that applying one of them makes the other applicable (undoing the effect of the first one).
In this case, the operation defined above guarantees that the database always reflects the \emph{last} actions that were executed.

We note also that $\UU_1\uplus\UU_2$ is consistent whenever $\UU_1$ and $\UU_2$ are.
%
\begin{definition}
  Let $\db$ be a database and $\eta$ be a set of AICs over $\db$.
  The operator $\Op^\db:2^\acts\to 2^\acts$ is defined as follows:
  \[
    \Op^\db(\UU) = \UU\uplus\{\head(r)\mid r\in \aics \land \UU(\db)\models\body(r)\}
  \]
\end{definition}
In other words, $\Op^\db(\UU)$ is obtained by updating $\UU$ with the heads of all AICs whose bodies are satisfied by $\UU(\db)$.
To see that this operator is well-defined (i.e., that the result of applying it indeed yields a consistent set of actions), observe that the syntactic restrictions on AICs guarantee that the set $\{\head(r)\mid r\in \aics \land \UU(\db)\models\body(r)\}$ is always consistent. 
 Indeed, if both $\add a$ and $\remove a$ are in this set, then there must be rules $r_1$ and $r_2$ such that $\neg a\in\body(r_1)$ and $a\in\body(r_2)$ with $\UU(\db)\models\body(r_i)$ for $i=1,2$, which is impossible since it would mean that $\UU(\db)\models a$ and $\UU(\db)\models \lnot a$.
 From this, it also follows that $\Op(\UU)$ is always consistent. 
As before, when $\db$ is clear from the context, we simply write $\Op$ for $\Op^\db$.


\begin{proposition}
 The operator $\Op$ is the only operator on $2^\acts$ that satisfies inertia, cancellation and completeness. 
\end{proposition}
\begin{proof}
 It is easy to verify that \Op also satisfies inertia, cancellation and completeness.
 

 Now, assume $O: 2^\acts \to 2^\acts$ satisfies the three properties. Let $\alpha$ be any action in \acts and $\UU\subseteq \acts$. 
 Let $\VV$ denote $\{\head(r)\mid r\in \aics \land \UU(\db)\models\body(r)\}$. 
 Since, $\alpha \in \acts$, we know that $\alpha^D\not \in \UU$. We show that $\alpha \in O(\UU)$ if and only if $\alpha \in \Op(\UU) = \UU\uplus \VV $. 
 We distinguish 3 cases:
 \begin{itemize}
  \item If $\alpha \in \VV$, then, since $\alpha^D\not\in \UU$, by completeness, $\alpha \in O(\UU)$. In this case, since $\alpha^D\not \in \UU\cup \VV$, but $\alpha \in \UU\cup\VV$, it holds that $\alpha \in \UU \uplus \VV = \Op(\UU)$. 
  \item If $\alpha^D \in \VV$, then there is a rule $r\in \aics$ such that $\alpha^D\in \head(r)$ and $\UU(\db)\models \body(r)$. Since $\UU(\db)\models \body(r)$, $\litof(\alpha)$ holds in $\UU(\db)$. Since $\alpha$ changes \db, we know that $\litof(\alpha)$ does not hold in $\db$, hence it must be the case that $\alpha \in \UU$. By cancellation, we then find that $\alpha\not\in O(\db)$. Since $\alpha \in \UU$ but $\alpha^D\in \VV$, also $\alpha\not\in \UU\uplus\VV=\Op(\UU)$.
  \item If neither $\alpha\in\VV$, nor $\alpha^D\in\VV$, then by inertia, it must hold that $\alpha \in O(\UU)$ if and only if $\alpha\in \UU$. In this case, it also holds that $\alpha \in \Op(\UU) = \UU \uplus\VV$ if and only if $\alpha \in \UU$.
  \end{itemize}
  Hence, we have proven in each of the cases that $\alpha \in O(\UU)$ if and only if $\alpha \in \Op(\UU)$ and the result follows. 
\end{proof}

% The resulting set may contain less update actions than the number of such rules, since different AICs may include the same action in their heads.


\begin{example}[Example \ref{ex:repairs} continued]
  Consider again the set of AICs $\eta$ from Example~\ref{ex:repairs}, where
  $\db=\{a,b\}$.
  Then $\Op(\emptyset)=\{\remove a, \remove b\}$. Indeed, the bodies of all rules are satisfied; hence all heads are elements of $\Op(\emptyset)$.
\end{example}

% \luis{this paragraph seems completely out of place. Delete?}
% \bart{I added/extended it since there was a reviewer concerned about this. It is important to note that this set is consistent to guarantee consisntency of $\Op^\db(\UU)$.}
% \luis{Yup, I thought that was the reason. But Proposition 5.4 already stated that it held, which was what confused me. I moved your text to inside that proof -- see if you agree.}
% % In the interest of legibility, we write simply $\Op$ instead of $\Op^\db$ whenever $\db$ is clear from the context.
% 
% The operator $\Op$ provides alternative characterizations of the notions of weak repair, repair, founded, well-founded and grounded sets of update actions.
% 
% \todo{RVWR 2:
% It looks to me that the operator $T^{DB}_{\eta}$ (that for simplicity here will be denotes as T)
% is not monotone (by definition). As a conseguence, not always a fixpoint can be reached.
% Consider the following AICs:
% 
% not b, a -> -a
% not b, not a -> +a
% 
% and an empty database DB.
% Then:
% 
% T(0)={+a}; T(T(0))={}; T(T(T(0)))={+a} ...
% 
% Observe that the {+b} is a repair.
% Could you explain better this point?}
% \bart{The reviewer seems to be confused between the difference of  ``being a fixpoint '' and ``being able to reach a fixpoint'' (by iterating starting from $\emptyset$). The ``being reachable'' is not relevant at this point. Still, we could add his theory as an example.}
% \luis{I'd rather address this in the reply, as it is such a strange comment. Anyway, I added a simple example about fixpoints not having to exist (which is a reasonable variant of what he says).}

\begin{proposition}
  \label{prop:weak-repair}
  Let $\db$ be a database, $\eta$ be a set of normal AICs over $\db$ and $\UU$ be a set of update actions.
  Then $\UU$ is a weak repair for $\fulldb$ if and only if $\UU$ is a fixpoint of $\Op$.
\end{proposition}
\begin{proof}
  If $\UU$ is a weak repair for $\fulldb$, then certainly, $\UU\in 2^{\acts}$. Also, $\UU(\db)\not\models\body(r)$ for all $r\in\eta$, whence $\Op(\UU)=\UU$.
  If $\UU$ is not a weak repair for $\fulldb$, then $\UU(\db)\models\body(r)$ for some $r\in\eta$, and $\Op(\UU)$ differs from $\UU$ by (at least) $\head(r)$.
\end{proof}

\begin{example}
  In general, $\Op$ does not need to have fixpoints (since there may be no database satisfying $\eta$).
  A simple (unrealistic) example is if $\eta$ consists of the two rules
  \[a\aicrule-a \qquad \mbox{ and } \qquad \neg a\aicrule +a\]
  where $\Op(\db)\neq\db$ for any database $\db$.
\end{example}

\begin{proposition}
  \label{prop:repair}
  Let $\db$ be a database, $\eta$ be a set of normal AICs over $\db$ and $\UU$ be a set of update actions.
  Then $\UU$ is a repair for $\fulldb$ if and only if $\UU$ is a minimal fixpoint of $\Op$.
\end{proposition}
\begin{proof}
  Follows directly from Proposition~\ref{prop:weak-repair} and the definition of repair.
\end{proof}

\begin{proposition}
  \label{prop:founded-char}
  Let $\db$ be a database, $\eta$ be a set of normal AICs over $\db$ and $\UU\in 2^{\acts}$. % be a consistent set of update actions.
  Then $\UU$ is founded with respect to $\fulldb$ if and only if, for all $\alpha\in\UU$, it is the case that $\alpha\in\Op(\UU\setminus\{\alpha\})$.
\end{proposition}
\begin{proof}
  It follows from the definition of \Op that for each action $\alpha\in \UU$, the following are equivalent:
  \begin{enumerate}
%    \item there is a rule $r\in\eta$ such that $\UU(\db)\models\body(r)\setminus\{\alpha^D\}$,
   \item there is a rule $r\in\eta$ such that $(\UU\setminus\{\alpha\})(\db)\models\body(r)$ and 
   \item $\alpha\in\Op(\UU\setminus\{\alpha\})$.
  \end{enumerate}
  Now, $\UU$ is founded if and only if \textbf{(i)} holds for all actions $\alpha\in \UU$; this is indeed equivalent with the condition from this proposition, which is that \textbf{(ii)} holds for each $\alpha\in \UU$. 
%   An action $\alpha\in\UU$ is founded if and only if there is a rule $r\in\eta$ such that $\UU(\db)\models\body(r)\setminus\{\alpha^D\}$.
%   This is equivalent to saying that $(\UU\setminus\{\alpha\})(\db)\models\body(r)$.
%   But, by definition of $\Op$, we have $\alpha\in\Op(\UU\setminus\{\alpha\})$ if and only if there is a rule $r\in\eta$ such that $(\UU\setminus\{\alpha\})(\db)\models\body(r)$, which concludes the proof.
% d
% %   Consistency of $\UU$ is needed for the direct implication, as $\alpha$ is only added to $\UU\setminus\{\alpha\}$ by $\Op$ if that set does not already contain $\alpha^D$. BART: NO: consistency is needed for concepts such as $\Op(\UU\setminus\{\alpha\})$ to be well-defined (i.e. for this lemma to make sense in the first place
\end{proof} 

This result also gives some intuition regarding why founded repairs allow for circular dependencies: the definition of founded repair only checks that each individual action is supported by the remaining ones, but it still allows for dependency cycles.

\begin{proposition}
  \label{prop:wf}
  Let $\db$ be a database, $\eta$ be a set of normal AICs over $\db$ and $\spaceUU$ be a weak repair for $\fulldb$.
  Then $\UU$ is well-founded if and only if there is an ordering $\alpha_1,\ldots,\alpha_n$ of the elements of\, $\UU$ such that $\alpha_i\in\Op(\{\alpha_1,\ldots,\alpha_{i-1}\})$ for each $i=1,\ldots,n$.
\end{proposition}
\begin{proof}
  If $\UU$ is well-founded, then the ordering is given by the sequence of actions introduced at each node in the path, in the well-founded repair tree for $\fulldb$, going from the root to the node with label $\UU$.
  Conversely, if $\UU$ can be obtained in the manner described, then it defines a valid path in that same tree ending at a leaf.
\end{proof}



\begin{proposition}\label{prop:grounded:ok}
 Let $\db$ be a database and $\eta$ a set of AICs over \db. A set of update actions $\spaceUU$ is a grounded repair of \fulldb if and only if $\spaceUU$ is a grounded fixpoint of \Op. 
\end{proposition}
\begin{proof}
Recall that grounded and strictly grounded fixpoints of \Op coincide.
We show that $\UU$ is a grounded repair of $\fulldb$ iff $\UU$ is a strictly grounded fixpoint of $\Op$.

First suppose
%\UU is a grounded repair of \fulldb. From Proposition \ref{lem:repair}, it follows that $\UU$ is a fixpoint of $\Op$. Suppose
that $\UU$ is not a strictly grounded fixpoint of \Op.
This means that there exists a set $\VV\subsetneq\UU$ such that $\Op(\VV)\cap\UU\subseteq\VV$.
From the definition of $\Op$ it follows immediately that, if $r$ is a rule such that $\VV(\db)\models\body(r)$, then $\head(r)\not\in(\UU\setminus\VV)$, whence $\UU$ is not a grounded repair for $\fulldb$.
 %This means that there exists a set of update actions $\VV$ such that $\Op(\UU\cap\VV) \subseteq\VV$ and $\UU \not \subseteq \VV$. 
%Taking $A=\UU\setminus V$, and $\UU'=\UU\setminus A$, we find that $\Op(\UU') \cap A = \emptyset$. This means (following the definition of \Op) that there is no rule whose body is satisfied  that derives an element of $A$ in $\UU'$.
% Since $\UU$ is a grounded repair, we arrive at a contradiction. 
 
Conversely, assume that $\UU$ is a strictly grounded fixpoint of \Op
% and let $A$ be any subset of $\UU$ such that no rule $r$ with $\head(r)\in A$ has $\body(r)$ satisfied in $\UU'=\UU\setminus A$. We should show that $A=\emptyset$. Now, from the definition of $\Op$, it follows that $\Op(\UU')\cap A=\emptyset$ and thus that $\Op(\UU') \cap \UU \subseteq \UU'$. Since groundedness and strict groundedness coincide, $\UU$ is strictly grounded and thus, $\UU'=\UU$ and $A=\emptyset$ indeed. 
and let $\VV$ be any subset of $\UU$ such that there is no rule $r$ for which $\VV(\db)\models\body(r)$ and $\head(r)\in(\UU\setminus\VV)$. From the definition of $\Op$, it follows that $\Op(\VV)\cap(\UU\setminus\VV)=\emptyset$ and thus that $\Op(\VV)\cap\UU\subseteq \VV$. Since $\UU$ is strictly grounded, it follows that $\VV=\UU$. 
\end{proof}

The previous proposition illustrates that Proposition \ref{prop:justified} is not a coincidence at all. 
Indeed, \mycitet{GroundedFixpoints} have already shown that all stable fixpoints of a given approximator are grounded, and 
\citet[Theorem~6]{tplp/CaropreseT11} showed that justified repairs are stable models of a given derived logic program. 
In the following sections, we explore this relationship further: first, we define an approximator for \Op and as such obtain also a notion of stable repair. 
Next, in Section \ref{sec:lp}, we  study the relationship between logic programs and AICs in depth. 

%\luis{From the commented out stuff, I would include the sentence about avoiding circularity of support and the theorem on complexity. What do you think? I deleted the rest.}
%\bart{Complexity comes later, together with the rest of complexity stuff}
%% \bart{Regarding the discussion of circularity of support. We do not give a formal definition of what this *actually* is. 
%% Our claim rests on the fact that ``no circularity of support'' would somehow coincide with ``grounded''. But this is hard to prove, given the lack of a definition of ``circularity''... }

%% \luis{I understand your point. My frustration is that Caroprese \&\ co.\ never do, either, but they somehow get away with claiming (with no proof) that justified repairs avoid it. The reason why I'd like to have a remark in this sense is to substantiate our claim that grounded repairs solve their problem too. (We hinted at this in footnote 1 on page 5.)}
%% \bart{I get your point, but to claim that we ``prove'' their claim is a bit too far for me (also in that footnote by the way). 
%% We can claim that our definition is simpler, directly inspired by the idea of not having circular support (whatever we take away, something must return), etcetera, but not that we prove some claim that has never been formally stated (without definition of what ``circularity'' is). Don't know yet how to phrase this nicely}

Since grounded repairs can be built from the ground up, this result also corroborates the informal claim that justified repairs avoid circularity of support, as stated by~\citet{tplp/CaropreseT11}.

\ignore{
The correspondence between justified repairs and answer sets for particular logic programs~\cite{Caroprese2011} shows that justified repairs can also be characterized in a related manner.
However, since answer sets of a logic program are models of its Gelfond--Lifschitz transform, the corresponding characterization in terms would be as fixpoints of the corresponding operator for a similarly derived set of AICs, rather than of $\Op$.

This result is not very surprising: justified weak repairs are answer sets of a particular logic program (Theorem 6 in~\cite{Caroprese2011}), and in turn answer sets of logic programs are grounded fixpoints of the consequence operator (see remark at the top of Section~5 in~\mycite{GroundedFixpoints}).
However, the translation defined in~\cite{Caroprese2011} is from logic programs to databases with AICs (rather than the other way around), so Proposition~\ref{prop:justified} is \emph{not} a direct consequence of those results.
Also, since grounded repairs can be built from the ground up, this result also establishes that justified repairs avoid circularity of support (a property that was claimed in~\cite{Caroprese2011}, but never formally discussed).


%%%%%%%%%%%%%%%%%%%%%%%%%


\begin{theorem}
  \label{thm:grounded-complexity}
  Let $\db$ be a database and $\eta$ be a set of normal AICs over $\db$.
  The problem of deciding whether there exist grounded repairs for $\fulldb$ is $\Sigma^P_2$-complete.
\end{theorem}
\begin{proof}
  For membership, we need to show that we can decide the problem with a non-deterministic Turing machine with an NP oracle.
  Given a set of update actions $\UU$, checking that it is a fixpoint of $\Op$ can be done in polynomial time on the size of $\db$ and $\eta$; the NP-oracle can then answer whether there exists $\UU'\subsetneq\UU$ with $\Op(\UU')\cap\UU\subseteq\UU'$, thereby establishing whether $\UU$ is grounded.

  For hardness, we invoke the (polynomial time) translation $\mathit{aic}$ from logic programs to sets of AICs over the empty database given Section~7 of~\cite{Caroprese2011}.
  Given a logic program $\mathcal P$, deciding whether $\langle\emptyset,\mathit{aic}(\mathcal P)\rangle$ has a grounded repair is equivalent to deciding whether $\mathcal P$ has a grounded model, which is $\Sigma^P_2$-complete by Theorem~5.7 of~\mycite{GroundedFixpoints}.
\end{proof}

This result still holds if we allow a truly first-order syntax for AICs, where the atoms can include variables that are implictly universally quantified.

%%% Local Variables:
%%% mode: latex
%%% TeX-master: "../AFT-semantics-AIC.tex"
%%% End:
}
