Founded, well-founded and justified repairs were all introduced with the purpose of characterizing a class of repairs whose actions are supported (there is a reason for having them in the set), and that support is not circular; in particular, these repairs should be constructible ``from the ground up'', which was the motivation for defining well-founded repairs.
However, all notions exhibit unsatisfactory examples: there exist founded repairs with circular support (see, e.g., Example \ref{ex:founded}), and repairs with no circular support that are not justified~\cite{tase/Cruz-FilipeEGN13}. %; well-founded repairs, on the other hand, are not stratifiable~\cite{lcf:14}, which impacts their computation in practice.
In this section, we introduce a new semantics, grounded repairs, aimed at directly tackling this issue. 

Grounded repairs are motivated by Example \ref{ex:founded}, where we noticed that the definition of founded repairs does not manage to capture groups of self-supporting arguments. 
Indeed, there the repair $\UU_1$ is founded. It consists of two actions, $\remove a$ and $\remove b$ such that whenever one of them is removed from $\UU_1$, there is an AIC whose body is violated and whose head is the action in question. 
However, if we remove \emph{both} of them simultaneously, we notice that no rule no longer applies. As such, we can conclude that these actions are ``self-supporting'': the only reason to have one of the two actions in the repair of our choice is because the other action is also in there. 
Our definition of grounded (weak) repair is aimed directly at avoiding this kind of situations. 

% \luis{I think the definitions look cleaner if we don't use $A$ -- I left the old ones commented out.}
\begin{definition}\label{def:grounded}
 A set of update actions \UU is \emph{grounded} with respect to $\fulldb$ if, for each $\VV \subsetneq  \UU$, there is a rule  $r\in \aics$ such that $\VV(\db)\models \body(r)$ and $\head(r)\in(\UU\setminus\VV)$. A \emph{grounded (weak) repair} is a (weak) repair that is grounded.
\end{definition}
%% \begin{definition}\label{def:grounded}
%%  A set of update actions \UU is \emph{grounded} with respect to $\fulldb$ if, for each $A \subseteq  \UU$, there is a rule  $r\in \aics$ such that $\UU'(\db)\models \body(r)$ and $\head(r)\in A$, where $\UU'=\UU\setminus A$. A \emph{grounded (weak) repair} is a (weak) repair that is grounded.
%% \end{definition}
  
As can be seen, our definition of groundedness is a slight variant of the notion of foundedness: instead of only considering what happens if \emph{one} action is dropped from a proposed set of update actions, we consider arbitrary removals. 
A first observation with respect to groundedness is that grounded weak repairs are always minimal, i.e., that each grounded weak repair is a repair. 
\begin{proposition}
 All grounded weak repairs of \fulldb are $\subseteq$-minimal, i.e., are repairs. 
\end{proposition}
\begin{proof}
 Suppose $\UU$ is a grounded weak repair and $\UU$ is not minimal, i.e., that there exists a $\VV\subsetneq \UU$ that is also a weak repair. 
% Now, take $A = \UU\setminus \VV$ in the definition of grounded. Then $\VV = \UU\setminus A$.
 Since $\UU$ is grounded, there must exist a rule whose body is satisfied in $\VV$, contradicting the fact  that $\VV$ is a weak repair. 
\end{proof}
Thus, the notion of groundedness intrinsically embodies the principle of minimality of change, unlike other kinds of weak repairs previously defined.
% Furthermore, grounded repairs also embody the notion of ``support'' previously defined.


\begin{proposition}
  \label{prop:founded}
  Let $\db$ be a database, $\eta$ be a set of AICs over $\db$ and $\UU$ be a grounded repair for $\fulldb$.
  Then $\UU$ is both founded and well-founded.
\end{proposition}
\begin{proof}
  Assume that $\UU$ is a grounded repair for $\fulldb$.
  The fact that $\UU$ is founded follows immediately from the definition of grounded repair, since for each action $\alpha$, $\VV=\UU\setminus\{\alpha\}$ is a strict subset of $\UU$. Hence, by groundedness of $\UU$, there must be a rule $r$ with $\head(r)\in \UU\setminus \VV=\{\alpha\}$, whose body is satisfied in $\VV$. We find that $\UU$ is founded indeed. % and for such $\VV$ the conditions of groundedness coincides with that of foundedness hence there must be a rule  $rby letting $A$ range over the singleton sets of actions in $\UU$.
  
  Now, we show how to construct a sequence of subsets of $\UU$ that illustrates that $\UU$ is well-founded. 
  For this sequence, we start from $\UU_0=\emptyset$ and construct $\UU_i=\UU_{i-1}\cup\{u_i\}$ by picking a rule $r\in \fulldb$ with $\head(r)=u_i\in\UU$ and $\UU_{i-1}(\db)\models \body(r)$.
  Since $\UU$ is grounded, if $\UU_{i-1}\subsetneq\UU$ then such a rule always exists,
  %(taking $A=\UU\setminus\UU_{i-1}$),
  and by construction $\UU_i\subseteq\UU$.
  But $\UU$ is finite, therefore this sequence converges towards $\UU$, and thus $\UU$ is a well-founded repair.
\end{proof}

However, the notion of groundedness repair is strictly stronger than both of these: Example~\ref{ex:founded}, presented earlier, also shows that some forms of circular justifications are avoided by grounded repairs.
\begin{example}[Example \ref{ex:founded} continued]
  \label{ex:founded-gr}
%   Recall the set of AICs from Example~\ref{ex:founded}, together with $\db=\{a,b\}$.
  The repair $\UU_1=\{-a,-b\}$ is a founded repair that is not grounded: taking
  %$A = \UU_1$, and thus
  $\VV=\emptyset$, we notice that no rule with $\remove a$ or $\add a$ in the head has its body satisfied in $\VV$. %satisfies $\Op(\VV)\cap\UU_1=\{+c\}\cap\UU_1=\emptyset\subseteq\VV$.
  The more natural repair $\UU_2=\{+c\}$ is also founded, and it is immediate to verify that it is also grounded.
\end{example}


Likewise, not all well-founded repairs are grounded. %, but not conversely.
% \luis{check example}
\begin{example}[Example \ref{ex:well-founded} continued]
  Consider again $\eta$ from Example~\ref{ex:well-founded}, with $\db=\emptyset$.
  As shown earlier, the two well-founded repairs for $\fulldb$ are $\UU_1=\{{+b},{+a}\}$ and $\UU_2=\{{+a},{+c}\}$.
  We already observed that $\UU_2$ is not founded, so it cannot be grounded; indeed,
  %taking $A = \{\add c\}, \VV=\UU\setminus A
  $\VV = \{{+a}\}$ is a set of update actions such that no rule $r$ with
  %$\head(r)\in A$ 
  $\head(r)\in(\UU\setminus\VV)$
  has its body satisfied in $\VV$. %$\Op(\UU')\cap\UU_2=\{{+a},{+b}\}\cap\UU_2=\emptyset\subseteq\UU'$.
\end{example}



We thus have that grounded repairs are always founded and well-founded; the next example shows that they do not correspond to the intersection of those classes.

\begin{example}\label{ex:founded:well-founded}
  Consider the following set of AICs $\eta$.
  \begin{align}
    \neg a,\neg b&\supset{+a} \label{rfwf13} \\
    a,\neg b&\supset{+b} \label{rfwf14} \\
    \neg a,b&\supset{-b} \label{rfwf15} \\
    a,b,\neg c&\supset{+c} \label{rfwf16} \\
    a,\neg b,c&\supset{+b} \label{rfwf17} \\
    \neg a,b,c&\supset{+a}\label{rfwf18}
  \end{align}
  Let $\db=\emptyset$.
  Then $\UU=\{{+a},{+b},{+c}\}$ is a repair for $\fulldb$: the first three constraints require ${+a}$ and ${+b}$ to be included in the database, and the last three state that no $2$-element subset of $\UU$ can be a repair.
  Furthermore, $\UU$ is founded (the last three rules also ensure that) and well-founded (starting with $\emptyset$, we are forced to apply rules \ref{rfwf13}, \ref{rfwf14} and \ref{rfwf16}, in that order).

  However, $\UU$ is not grounded: if $\VV=\{{+b}\}$, then $\VV\subsetneq\UU$, but there is no rule $r$ with $\head(r)\in\{\add a, \add c\} $ whose body is satisfied in $\VV$. %-$\Op(\UU')\cap\UU=\emptyset\cap\UU=\emptyset\subseteq\UU'$.
\end{example}
In this situation, $\UU$ might seem reasonable; however, observe that the support for its actions \emph{is} circular: it is the three last rules that make $\UU$ founded, and none of them is applicable to $\db$.
Also note that $\VV(\db)=\{b\}$ is a database for which the given set $\eta$ behaves very awkwardly: the only applicable AIC tells us to remove $b$, while the only repair of $\VV(\db)$ is actually $\{{+a},{+c}\}$.

We do not feel that this example weakens the case for studying ground repairs, though: the consensual approach to different notions of repair is that they express \emph{preferences}.
In this case, where $\fulldb$ admits no grounded repair, it is sensible to allow a repair in a larger class -- and a repair that is both founded and well-founded is a good candidate.
The discussion by \citet[Section 8]{tplp/CaropreseT11} already proposes such a ``methodology'': choose a repair from the most restrictive category (justified, founded, or any).
We advocate a similar approach, but including grounded repairs among the possible choices.
% \bart{quite strong preferences expressed here... I would prefer to not be so explicit about which ones to prefer. Also, I find well-founded repairs very counterintuititve... }
% \luis{better like this? :-)} OK


We now investigate the relation between grounded and justified repairs, and find that all justified repairs are grounded, but not conversely -- in line with our earlier claim that the notion of justified repair is too strong.

\begin{proposition}
  \label{prop:justified}
  Let $\db$ be a database, and let $\eta$ be a set of normal AICs over $\db$. If $\UU$ is a justified repair for $\fulldb$, then $\UU$ is grounded.
\end{proposition}
\begin{proof}
  Let $\UU$ be a justified repair for $\fulldb$ and take $\VV\subsetneq\UU$.
  Then $\VV\cup\neff(\UU)$ is not closed under $\eta$, whence there is a rule $r\in\eta$ such that $\actof(\nup(r))\subseteq\VV\cup\neff(\UU)$ and $\head(r)\not\in\VV\cup\neff(\UU)$.

  Since $\VV\subseteq\UU$, also $\actof(\nup(r))\subseteq\UU\cup\neff(\UU)$, whence $\head(r)\in\UU\cup\neff(\UU)$ as $\UU$ is closed under $\eta$.
  But $\head(r)\not\in\VV\cup\neff(\UU)$, so $\head(r)\in\UU\setminus\VV$.

  We need to show that also $\VV\models\body(r)$. On the one hand, $\actof(\nup(r))\subseteq\VV\cup\neff(\UU)$ implies that $\VV(\db)\models\nup(r)$, as $\neff(\UU)\subseteq\neff(\VV)$; on the other hand, from $\head(r)\in\UU$ we know that $\litof(\head(r))^D\in\db$ (all actions in $\UU$ change $\db$), whence $\VV(\db)\models\litof(\head(r)^D)$ since $\head(r)\not\in\VV$.
  As $r$ is normal, there are no more literals in $\body(r)$, so $\VV(\db)\models\body(r)$. 
  Hence, we have found a rule $r$ such that $\VV\models \body(r)$ and $\head(r)\in(\UU\setminus\VV)$, thus showing that $\UU$ is grounded.
\end{proof}
%% \begin{proof}
%%  Assume $\UU$ is a justified repair.
%%   Take an arbitrary $A \subsetneq \UU$ and let $\UU'=\UU\setminus A$. 
%%   Then $\UU'\cup\neff(\UU)$ is not closed under $\eta$, whence there is a rule $r\in\eta$ such that $\actof(\nup(r))\subseteq\UU'\cup\neff(\UU)$ and $\head(r)\not\in\UU'\cup\neff(\UU)$.

%%   Since $\UU'\subseteq\UU$, also $\actof(\nup(r))\subseteq\UU\cup\neff(\UU)$, whence $\head(r)\in\UU\cup\neff(\UU)$ as $\UU$ is closed under $\eta$.
%%   But $\head(r)\not\in\UU'\cup\neff(\UU)$, so $\head(r)\in\UU\setminus\UU' = A $.

%%   We need to show that also $\UU'\models\body(r)$. On the one hand, $\actof(\nup(r))\subseteq\UU'\cup\neff(\UU)$ implies that $\UU'(\db)\models\nup(r)$, as $\neff(\UU)\subseteq\neff(\UU')$; on the other hand, from $\head(r)\in\UU$ we know that $\litof(\head(r))^D\in\db$ (all actions in $\UU$ change $\db$), whence $\UU'(\db)\models\litof(\head(r)^D)$ since $\head(r)\not\in\UU'$.
%%   As $r$ is normal, there are no more literals in $\body(r)$, so $\UU'(\db)\models\body(r)$. 
%%   Hence, we have found a rule $r$ with $\head(r)\in A$, and $\UU'\models \body(r)$ which we needed in order to show that $\UU$ is grounded.
%% \end{proof}
This proof does not use the hypothesis that $\UU$ is a repair.
This is due to the fact (already mentioned earlier) that we only consider \emph{normal} AICs in this paper. By Theorem~4 of~\cite{tplp/CaropreseT11}, all justified weak repairs are minimal when $\eta$ consists of only normal AICs.

%%%%%%%%%%%%%%%%%%%%%%%%

Recall Example~\ref{ex:justified}, which was used by \citet{tase/Cruz-FilipeEGN13} to point out that justified repairs sometimes eliminate ``natural'' repairs.
This example also shows that the notion of justified repair is stricter than that of grounded repair.
\begin{example}[Example~\ref{ex:justified} continued]
%   Recall the set $\eta$ from Example~\ref{ex:justified}, with $\db=\{a,b\}$.
  Although the repair $\UU=\{-a,-b\}$ for $\fulldb$ is not justified, it is grounded: if ${-a}\in\VV\subsetneq\UU$, then there is a rule that derives $\remove b$; otherwise, there is a rule that derives $\remove a$. %$\Op(\UU')\cap\UU$ contains ${-b}\in\UU\setminus\UU'$, else $\Op(\UU')\cap\UU$ contains ${-a}\in\UU\setminus\UU'$.
\end{example}
As discussed earlier, in this case the first rule clearly motivates the action $-a$, and the last rule then requires $-b$.
This is in contrast to Example~\ref{ex:founded}, where there was no clear reason to include either $-a$ or $-b$ in a repair.
Hence grounded repairs avoid this type of unreasonable circularities, without being as restrictive as justified repairs.

We summarize the relations between the different types of repairs in
Figure~\ref{fig:repairs}.
%
\begin{figure}
  \centering
  \def\svgwidth{.5\columnwidth}
  \import{files/}{repairs.pdf_tex}
  \caption{Relative inclusions between the sets of founded ($\mathcal F$), well-founded ($\mathcal{WF}$), grounded ($\mathcal G$) and justified ($\mathcal J$) repairs.
  All inclusions are strict.}
  \label{fig:repairs}
\end{figure}
%% \[\xymatrix@R-1em@C+2em{%
%%   \mathcal F \ar@{-}[dd]|{\neq} \ar@{-}[dr]|{\subsetneq} \\
%%   & \mathcal G \ar@{-}[r]|{\subsetneq} & \mathcal J \\
%%   \mathcal{WF} \ar@{-}[ur]|{\subsetneq}
%% }
%% \]

%%% Local Variables: 
%%% mode: latex
%%% TeX-master: "../AFT-semantics-AIC"
%%% End: 
