Founded, well-founded and justified repairs were all introduced with the purpose of characterizing a class of repairs whose actions are supported (there is a reason for having them in the set), and that support is not circular; in particular, these repairs should be constructible ``from the ground up'', which was the motivation for defining well-founded repairs.
However, all notions exhibit unsatisfactory examples: there exist founded repairs with circular support (see, e.g., Example \ref{ex:founded}) and repairs with no circular support that are not justified~\cite{CEGN13}. %; well-founded repairs, on the other hand, are not stratifiable~\cite{lcf:14}, which impacts their computation in practice.
In this section, we introduce a new semantics, grounded repairs, aimed at directly tackling this issue. 

Grounded repairs are motivated by Example \ref{ex:founded}, where we noticed that the definition of founded repairs does not manage to capture groups of self-supporting arguments. 
Indeed, there the repair $\UU_1$ is founded. It consists of two actions, $\remove a$ and $\remove b$ such that whenever one of them is removed from $\UU_1$, there is an AIC whose body is violated and whose head is the action in question. 
However, if we remove both of them simultaneously, we notice that no rule no longer applies. As such, we might conclude that these actions are ``self-supporting'': the only reason to have one of the two actions in the repair of our choice is because the other action is also in there. 
Our definition of grounded (weak) repair is aimed directly at avoiding this kind of situations. 

\begin{definition}\label{def:grounded}
 A set of update actions \UU is \emph{grounded} with respect to $\fulldb$ if, for each $A \subseteq  \UU$, there is a rule  $r\in \aics$ with $\head(r)\in A$ and such that $\UU'(\db)\models \body(r)$, where $\UU'=\UU\setminus A$. A \emph{grounded (weak) repair} is a (weak) repair that is grounded.
 \end{definition}
  
As can be seen, our definition of groundedness is a slight variant of the notion of foundedness: instead of only considering what happens if \emph{one} action is dropped from a proposed set of update actions, we consider arbitrary removals. 
A first observation with respect to groundedness is that grounded weak repairs are always minimal, i.e., that each grounded weak repair is a repair. 
\begin{lemma}
 All grounded weak repairs of \fulldb are minimal. 
\end{lemma}
\begin{proof}
 Suppose $\UU$ is a grounded weak repair and $\UU$ is not minimal, i.e., that there exists a $\VV\subsetneq \UU$ that is also a weak repair. 
 Now, take $A = \UU\setminus \VV$ in the definition of grounded. Then $\VV = \UU\setminus A$. Since $\UU$ is grounded, there must exist a rule whose body is satisfied in $\UU$. However, this contradicts the fact  that $\VV$ is a weak repair. 
\end{proof}
Thus, the notion of grounded repair intrinsically embodies the principle of minimality of change, unlike other kinds of weak repairs previously defined.
Furthermore, grounded repairs also embody the notion of ``support'' previously defined.



\begin{lemma}
  \label{lem:founded}
  Let $\db$ be a database, $\eta$ be a set of AICs over $\db$ and $\UU$ be a grounded repair for $\fulldb$.
  Then $\UU$ is both founded and well-founded.
\end{lemma}
\begin{proof}
  Assume that $\UU$ is a grounded repair for $\fulldb$.
  The fact that $\UU$ is founded follows immediately from the definitions of founded and grounded repairs. 
  
%   
%   For each $\alpha\in\UU$, necessarily 
%   
%   
%   $\Op(\UU\setminus\{\alpha\})\cap\UU\subsetneq(\UU\setminus\{\alpha\})$, which implies that $\alpha\in\Op(\UU\setminus\{\alpha\})$.
%   By Lemma~\ref{lem:founded-char}, this implies that $\UU$ is founded.

  Now, we show how to construct a sequence that illustrates that $\UU$ is well-founded. 
  For this sequence, we start from $\UU_0=\emptyset$ and always set one $\UU_i = \UU_{i_1}\cup \{u_i\}$, where $u_i\in \UU$ is an arbitrary update action such that there is a rule $r\in \fulldb$ with $\head(r) = u_i$ and $\UU_{i-1}(\db)\models \body(r)$. By the definition of groundedness (taking $A=\UU\setminus \UU_{i-1}$) such $u_i$ always exists as long as $\UU_{i-1} \subsetneq \UU$. Hence, this sequence converges towards $\UU$ and indeed, $\UU$ is a well-founded repair. 
%   
%   
%   until this is no longer possible.
%   Letting $\UU'$ be the last constructed set, by construction both $\UU'\subseteq\UU$ and $\Op(\UU')\cap\UU\subseteq\UU'$ (otherwise we could proceed).
%   Since $\UU$ is grounded, it cannot be the case that $\UU'\subsetneq\UU$, so $\UU=\UU'$ and it is thus a well-founded repair.
\end{proof}


\luis{check example, add new one here}
However, the notion of grounded repair is strictly stronger than both of these: Example~\ref{ex:founded}, presented earlier, also shows that some forms of circular justifications are avoided by grounded repairs.
\begin{example}[Example \ref{ex:founded} continued]
  \label{ex:founded-gr}
%   Recall the set of AICs from Example~\ref{ex:founded}, together with $\db=\{a,b\}$.
  The repair $\UU_1=\{-a,-b\}$ is a founded repair that is not grounded: taking $A = \UU$ and thus $\UU'=\emptyset$, we notice that no rule with $\remove a$ or $\add a $ in the head has a body satisfied in $\UU'$. %satisfies $\Op(\UU')\cap\UU_1=\{+c\}\cap\UU_1=\emptyset\subseteq\UU'$.
  The more natural repair $\UU_2=\{+c\}$ is also founded, and it is immediate to verify that it is also grounded.
\end{example}


Likewise, not all well-founded repairs are grounded. %, but not conversely.
% \luis{check example}
\begin{example}[Example \ref{ex:well-founded} continued]
  Consider again $\eta$ from Example~\ref{ex:well-founded}, with $\db=\emptyset$.
  As shown earlier, the two well-founded repairs for $\fulldb$ are $\UU_1=\{{+b},{+a}\}$ and $\UU_2=\{{+a},{+c}\}$.
  We already observed that $\UU_2$ is not founded, so it cannot be grounded; indeed, taking $A = \{\add c\}, \UU'=\UU\setminus A = \{{+a}\}$ is a set of update actions such that no rule $r$ with $\head(r)\in A$ has its body satisfied in $\UU'$. %$\Op(\UU')\cap\UU_2=\{{+a},{+b}\}\cap\UU_2=\emptyset\subseteq\UU'$.
\end{example}



We thus have that grounded repairs are always founded and well-founded; the next example shows that they do not correspond to the intersection of those classes.

\begin{example}\label{ex:founded:well-founded}
  Consider the following set of AICs $\eta$.
  \begin{align}
    \neg a,\neg b&\supset{+a} \label{rfwf13}) \\
    a,\neg b&\supset{+b} \label{rfwf14}) \\
    \neg a,b&\supset{-b} \label{rfwf15}) \\
    a,b,\neg c&\supset{+c} \label{rfwf16}) \\
    a,\neg b,c&\supset{+b} \label{rfwf17}) \\
    \neg a,b,c&\supset{+a}\label{rfwf18})
  \end{align}
  Let $\db=\emptyset$.
  Then $\UU=\{{+a},{+b},{+c}\}$ is a repair for $\fulldb$: the first row of constraints requires ${+a}$ and ${+b}$ to be included in the database, and the second row states that no $2$-element subset of $\UU$ can be a repair.
  Furthermore, $\UU$ is founded (the rules in the second row ensure that) and well-founded (starting with $\emptyset$, we are forced to apply rules \ref{rfwf13}, \ref{rfwf14} and \ref{rfwf16}, in that order).

  However, $\UU$ is not grounded for $\Op$: if $\UU'=\{{+b}\}$, then $\UU'\subsetneq\UU$, but there is no rule $r$ with $\head(r)\in\{\add a, \add c\} $ whose body is satisfied in $\UU'$. %-$\Op(\UU')\cap\UU=\emptyset\cap\UU=\emptyset\subseteq\UU'$.
\end{example}
In this situation, $\UU$ might seem reasonable; however, observe that the support for its actions \emph{is} circular: it is the three rules in the second row that make $\UU$ founded, and none of them is applicable to $\db$.
Also note that $\UU'(\db)$ is a database for which the given set $\eta$ behaves very awkwardly: the only applicable AIC tells us to remove $b$, but the only possible repair is actually $\{{+a},{+c}\}$.

We do not feel that this example weakens the case for studying ground repairs, though: the consensual approach to different notions of repair is that they express \emph{preferences}.
In this case, where $\fulldb$ admits no grounded repair, it is sensible to allow a repair in a larger class -- and a repair that is both founded and well-founded is a good candidate.
The discussion in Section~8 of~\cite{tplp/CaropreseT11} already proposes such a ``methodology'': choose a repair from the most restrictive category (justified, founded, or any).
We advocate a similar approach, but dropping justified repairs in favor of grounded repairs, and preferring well-founded to founded repairs.
\bart{quite strong preferences expressed here... I would prefer to not be so explicit about which ones to prefer. Also, I find well-founded repairs very counterintuititve... }


We now investigate the relation to justified repairs, and find that all justified repairs are grounded, but not conversely -- confirming our earlier claim that the notion of justified repair is too strong.

\begin{lemma}
  \label{lem:justified}
  Let $\db$ be a database, and let $\eta$ be a set of normal AICs over $\db$. If $\UU$ is a justified repair for $\fulldb$, then $\UU$ is grounded.
\end{lemma}
\begin{proof}
 Assume $\UU$ is a justified repair.. 
  Take an arbitrary $A \subsetneq \UU$ and let $\UU'=\UU\setminus A$. 
%   Let $\UU$ be a justified repair for $\fulldb$ and assume that $\UU'\subsetneq\UU$.
  Then $\UU'\cup\neff(\UU)$ is not closed under $\eta$, whence there is a rule $r\in\eta$ such that $\actof(\nup(r))\subseteq\UU'\cup\neff(\UU)$ and $\head(r)\not\in\UU'\cup\neff(\UU)$.

  Since $\UU'\subseteq\UU$, also $\actof(\nup(r))\subseteq\UU\cup\neff(\UU)$, whence $\head(r)\in\UU\cup\neff(\UU)$ as $\UU$ is closed under $\eta$.
  But $\head(r)\not\in\UU'\cup\neff(\UU)$, so $\head(r)\in\UU\setminus\UU' = A $.

  We need to show that also $\UU'\models\body(r)$. On the one hand, $\actof(\nup(r))\subseteq\UU'\cup\neff(\UU)$ implies that $\UU'(\db)\models\nup(r)$, as $\neff(\UU)\subseteq\neff(\UU')$; on the other hand, from $\head(r)\in\UU$ we know that $\litof(\head(r))^D\in\db$ (all actions in $\UU$ change $\db$), whence $\UU'(\db)\models\litof(\head(r)^D)$ since $\head(r)\not\in\UU'$.
  As $r$ is normal, there are no more literals in $\body(r)$, so $\UU'(\db)\models\body(r)$. 
  Hence, we have found a rule $r$ with $\head(r)\in A$, and $\UU'\models \body(r)$ which we needed in order to show that $\UU$ is grounded. %and therefore $\head(r)\in\Op(\UU')$.
% 
%   We thus conclude that $\Op(\UU')\cap\UU\not\subseteq\UU'$.
%   By arbitrariness of $\UU'$, it follows that $\UU$ is grounded.
\end{proof}
This proof may look suspicious, as it does not use the hypothesis that $\UU$ is a repair.
However, this is due to the fact that we only consider \emph{normal} AICs in this paper. Indeed, Theorem~4 of~\cite{tplp/CaropreseT11} shows that all justified weak repairs are minimal when $\eta$ contains only normal AICs.

%%%%%%%%%%%%%%%%%%%%%%%%ùù



Recall Example~\ref{ex:justified}, which was used by \citet{tase/Cruz-FilipeEGN13} to point out that justified repairs sometimes eliminate ``natural'' repairs.
This example also shows that the notion of justified repair is also stricter than that of grounded repair.
\begin{example}[Example~\ref{ex:justified} continued]
%   Recall the set $\eta$ from Example~\ref{ex:justified}, with $\db=\{a,b\}$.
  Although the repair $\UU=\{-a,-b\}$ for $\fulldb$ is not justified, it is grounded: if ${-a}\in\UU'\subsetneq\UU$, then there is a rule that derives $\remove b$; otherwise, there is a rule that derives $\remove a$. %$\Op(\UU')\cap\UU$ contains ${-b}\in\UU\setminus\UU'$, else $\Op(\UU')\cap\UU$ contains ${-a}\in\UU\setminus\UU'$.
\end{example}
As discussed earlier, in this case the first rule clearly motivates the action $-a$, and the last rule then requires $-b$.
This is in contrast to Example~\ref{ex:founded}, where there was no clear reason to include either $-a$ or $-b$ in a repair.
Hence grounded repairs avoid this type of unreasonable circularities, without being as restrictive as justified repairs.

We summarize the relations between the different types of repairs in the following diagram.
\[\xymatrix@R-1em@C+2em{%
  \mathcal F \ar@{-}[dd]|{\neq} \ar@{-}[dr]|{\subsetneq} \\
  & \mathcal G \ar@{-}[r]|{\subsetneq} & \mathcal J \\
  \mathcal{WF} \ar@{-}[ur]|{\subsetneq}
}
\]
\luis{referee suggested Venn chart, I have one somewhere}
\bart{Add venn chart or legend}







