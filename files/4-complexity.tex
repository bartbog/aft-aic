\todo{Some complexity results.
Probably: well-founded/KK: polynomially computable (data complexity), stable fixpoints: NP-complete, ... Copy results from logic programming basically. Similarly, complexity increases if we start using ultimate approximators}

\luis{The plan: recall the translation $\mathsf{aic}$ from the Caroprese paper, show that the operator we use to define fixpoints is preserved by this operator, get upper bounds for all complexity problems.  For lower bounds I need to think a bit more.}

\bart{Lower bound for stable: every normal logic program can be seen  as an AIC. Should have te same stable smeantics (note: our transformation in theprevious section adds rules 
\[a\lrule a\] but these do not change stable models. 

Lower bounds for KK well-founded. Well...polytime upper bound is al that needs to be sshown?}
%%% Local Variables:
%%% mode: latex
%%% TeX-master: "../AFT-semantics-AIC.tex"
%%% End:
