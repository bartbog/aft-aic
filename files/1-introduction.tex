Modern databases are often equipped with integrity constraints that specify which are valid states of the database. 
Sometimes, databases can reach states in which one of the integrity constraints is violated, for instance due to execution of some invalid update, due to merging two databases, ... 
In such case, one needs to repair the database \cite{ai/EiterG92}. 
The problem with this is that it is often unclear which repairs should be considered ``good repairs''. 
This is where active integrity constraints (AICs)  kick in \mycite{AIC} . 
The idea of AICs is that instead of simply specifying which properties a valid state of the database should adhere to, one also specifies how violation of these properties can be repaired. 
However, when there are multiple active integrity constraints present, there may be dependencies between them. 
For instance, certain ways of fixing one AIC might result in a state where another AIC is violated. 
Defining a natural semantics for such AICs has been a topic of intensive research \cite{iclp/CaropreseGSZ06,sebd/MolinaroGC06,iclp/CaropreseTZ07,dasfaa/CaropreseGM07,iclp/CaropreseT08,tkde/CaropreseGZ09,tplp/CaropreseT11,corr/Cruz-Filipe16}. 

It is striking that many of the intuitions underlying about what ``good'' repairs are, are very similar to intuitions that have appeared in other domains of non-monotonic reasoning, such as for instance logic programming \mycite{LP} or default logic \mycite{DL}. 
For instance \todo{minimality} of change \cite{}, ... 
Still, it has been hard to find a set of satisfying set of semantics for AICs: this goes wrong that goes wrong. 
Essentially, fixing similar problems as in other NMR domains... 


In this paper, we define a new class of semantics for AICs that are natural counterparts of existing semantics in  various non-monotonic reasoning domains. 
To be precise, we define semantics for AICs based on approximation fixpoint theory \mycite{AFT}, an abstract algebraic framework that unifies semantics of logic programming, default logic, autoepistemic logic \mycite{AEL}, abstract argumentation frameworks \mycite{AF} and abstract dialectical frameworks \mycite{ADF}. 
In order to do so, we continue the work of Cruz-Filipe \cite{corr/Cruz-Filipe16}, who defined a semantic operator for AICs and used it to study the grounded fixpoint semantics \mycite{GroundedFixpoints} for AICs. 
In this paper, we define an approximating operator of Cruz-Filipe's semantic operator. 
When this operator is defined, approximation fixpoint theory immediately provides us with \textit{(i)} a supported fixpoint semantics, \textit{(ii)} a well-founded fixpoint semantics, \textit{(iii)} a (partial) stable fixpoint semantics, \textit{(iv)} a Kripke-Kleene fixpoint semantics, and \textit{(v)} a (partial) grounded fixpoint semantics. 
We study properties of these new semantics and study how they compare to existing semantics. 
Furthermore, we argue that from a practical point of view, the AFT-style well-founded semantics is very valuable. 
Indeed, we show that this semantics can be computed in polynomial time and that on a broad set of practical examples, it yields the intended result, providing an upper and lower bound on the set of acceptable repairs.  

Somewhere we should write:
\begin{compactitem}
 \item Our work provides better insights in the relationship between logic programming (or AEL, DL, ADF, AF) and AICs
\item We define a new set of semantics with desirable properties:
\begin{compactitem}
\item  Shifting
 \item WFF: good over and under approximation of what repairs should look like
 \item WFF: polytime
 \item SS: very close to grounded fixpoints, but computationally more attractive
\end{compactitem}\item Our work opens the way to applying a rich algebraic theory to AICs: stratification and modularity results \cite{tocl/VennekensGD06,tocl/BogaertsVD16} and predicate introduction results \cite{VennekensMWD07a,VennekensMWD07b}. 
\end{compactitem}
 