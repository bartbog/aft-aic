%\todo{RVWR1: Though I can picture how AIGs might be used in practical database management, I still wish to see a motivating example for practical applications.}
%\bart{We should make a ``real-life''-like example of a database, with AICs or AIGs.}

One of the key components of modern-day databases are integrity constraints: logical formulas that specify semantic relationships between the data being modeled that have to be satisfied at all times.
When the database is changed (typically due to updating), it is necessary to check if its integrity constraints still hold; in the negative case, the database must be repaired.

The problem of database repair has been an important topic of research for more than thirty years~\cite{icdt/Abiteboul88}.
There are two major problems when deciding how to repair an inconsistent database: \emph{finding} possible repairs and \emph{choosing} which one to apply.
Indeed, there are typically several ways to fix an inconsistent database, and several criteria to choose the ``best'' one have been proposed over the years.
Among the most widely accepted criteria are minimality of change~\cite{Winslett90,ai/EiterG92} -- change as little as possible -- and the common-sense law of inertia~\cite[discussed in, e.g.,][]{jlp/PrzymusinskiT97} -- do not change anything \longpaper{unless there is a reason for the change.}\shortpaper{without a reason to do it.}

A typical implementation of integrity constraints in database systems is by means of event-condition-action (ECA) rules~\cite{vldb/TenienteO95,mk/WidomC96}, which specify update actions to be performed when a particular event (a trigger) occurs and specific conditions hold.
ECA rules are widely used in practice, as they are simple to implement and their individual semantics is easy to understand.
However, the lack of declarative semantics for ECA rules makes their interaction complex to analyze and their joint behavior hard to understand.

The formalism of \emph{active integrity constraints (AICs)} \cite{ppdp/FlescaGZ04} was inspired by a similar idea.
AICs express database dependencies through logic programming-style rules that include update actions in their heads.
They come with a set of declarative semantics that identifies several progressively more restricted classes of repairs, which can be used as criteria to select a preferred repair~\cite{tplp/CaropreseT11}.
These repairs can be computed \longpaper{directly }by means of tree algorithms\longpaper{~\cite{tase/Cruz-FilipeGEN13}, which have been implemented as a prototype~\cite{ic3k/Cruz-FilipeFHLN15}.}\shortpaper{\cite{tase/Cruz-FilipeGEN13,CFHLNS15}.}

\begin{example}\label{ex:intro}
  We motivate the use of AICs in practice by means of a simple example.
  Consider a company's database, including tables $\mathrm{employee}$ and $\mathrm{dept}$ (relating employees to the department where they work).
  In particular, each employee is assigned to a unique department; if an employee is listed as working in two different departments, then the database is inconsistent, and this inconsistency must be fixed by removing one of those entries.

  We can write this requirement as the following AIC.
  \[
  \forall x, y, z: \mathrm{employee}(x),\mathrm{dept}(x,y),\mathrm{dept}(x,z),y\neq z \aicrule -\mathrm{dept}(x,y)
  \]

  The intended meaning of this rule is: if all the literals in the lefthandside (body) of the rule are true in some state of the database, for particular values of $x$, $y$ and $z$, then the database is inconsistent, and this inconsistency can be solved by performing the action on the right.

  Suppose that the database is
  \[\db=\{\mathrm{employee}(\mathrm{john}),\mathrm{dept}(\mathrm{john},\mathrm{finance}),\mathrm{dept}(\mathrm{john},\mathrm{hr})\}\,.\]
  This database is inconsistent, and applying our AIC with $x=\mathrm{john}$, $y=\mathrm{finance}$ and $z=\mathrm{hr}$ gives us a possible fix consisting of the action ``remove $\mathrm{dept}(\mathrm{john},\mathrm{finance})$''.
  Observe, however, that the instantiation $x=\mathrm{john}$, $y=\mathrm{hr}$ and $z=\mathrm{finance}$ detects the same inconsistency, but proposes instead the fix ``remove $\mathrm{dept}(\mathrm{john},\mathrm{hr})$'': in general, there can be several different ways to repair inconsistencies.

  AICs may also interact with each other.
  Suppose that we add the constraint
  \begin{align} \forall x, y, z: \mathrm{supervisor}(x,y),\mathrm{dept}(x,z),\neg\mathrm{dept}(y,z)\aicrule +\mathrm{dept}(y,z)\label{rule:dept}\end{align}
  stating that employees can only supervise people from their own department, and that whenever this constraint is violated, the department of the supervisee needs to be updated (i.e., the  $\mathrm{supervisor}$ table and the department of the supervisor are deemed correct. 
  If the database is now
  \[
    \db=\{\mathrm{employee}(\mathrm{john}),\mathrm{employee}(\mathrm{ann}),
    \mathrm{dept}(\mathrm{john},\mathrm{finance}),\mathrm{dept}(\mathrm{ann},\mathrm{hr}),
    \mathrm{supervisor}(\mathrm{ann},\mathrm{john})\}
  \]
  then this AIC detects an inconsistency, and suggests that it be fixed by adding the entry $\mathrm{dept}(\mathrm{john},\mathrm{hr})$.
  The database is still inconsistent, though, since there are now two entries for John in the $\mathrm{dept}$ table; restoring inconsistency would also require removing the entry $\mathrm{dept}(\mathrm{john},\mathrm{finance})$.

  An alternative repair of the integrity constraint that the supervisee and supervisor should belong to the same department would be to change the department information associated with $\mathrm{ann}$. 
  By using \emph{active} integrity constraints, we discard this solution: rule \eqref{rule:dept} only allows to insert a new department for the supervisee. 
  If we additionally also want to allow changing $\mathrm{ann}$'s department, we need an extra constraint. 
%   Alternatively, the database could be repaired by instead changing the department information for Ann. 
% %   However, such a repair will be disallowed by the semantics of AICs, as 
%   The semantics of AICs make it clear that the former solution is preferable.
\end{example}	

%% \todo{here example real-life? E.G., company database. A constraint that no person can be a member both of the department finance \& accounting and the department human resources. If such a situation occurs, remove membership of HR. 

%% While I was writing this example (thinking about it), I stumbled onto some restrictions of AICs. For instance:
%% ``No one can have a salary greater thann 100.000. If someone does, cap it to 100.000'' 
%% You cannot write this in AICs. } 
%% \luis{There are some simple extensions where you can :) unfortunately that paper has been extremely hard to get published :/ we can discuss this.}
%% \bart{I'ld love te hear it}
%% \luis{Essentially you allow the bodies of rules to use also extra predicates with fixed semantic, like decidable fragments of arithmetic. For the heads, you can disallow actions involving those predicates. Your example then becomes something like:
%% \begin{align*}
%% \mbox{salary}(X,Y),Y>100000,\neg\mbox{fixSalary}(X) & \aicrule +\mbox{fixSalary}(X) \\
%% \mbox{salary}(X,Y),Y>100000,\mbox{fixSalary}(X) & \aicrule -\mbox{salary}(X,Y) \\
%% \mbox{fixSalary}(X),\neg\mbox{salary}(X,Y) & \aicrule +\mbox{salary}(X,100000) \\
%% \mbox{salary}(X,Y),Y\leq100000,\mbox{fixSalary}(X) & \aicrule -\mbox{fixSalary}(X)
%% \end{align*}
%% In the third rule, you also use universal quantification on a free variable, which is ok if the variable is a singleton. Nevertheless, the referees of that paper said that the contributions were non-existent, so we're still trying to get it published somewhere :-)
%% }
%% \bart{About the third rule. Let's be clear: the meaning is: 
%% \begin{align*}
%%  \mbox{fixSalary}(X),(\neg\exists Y:\mbox{salary}(X,Y)) & \aicrule +\mbox{salary}(X,100000) \\
%% \end{align*}
%% right? 
%% I would always make this explicit. Gets confusing otherwise. 
%% What your write now is not really what I had in mind. It is very procedural, among others because of the ``temporary'' variable fixsalary. It's not really nice KR. 
%% It feels like a workaround. 
%% For instance, none of our semantics will work for this kind of example. (because of that fixSalary)
%% I would like to be able to write something like this
%% \begin{align*}
%% \mbox{salary}(X) > 100.000 \aicrule \mbox{salary}(X) = 100.000
%% \end{align*}
%% I.e/; I especially need functions! 

%% Regarding the reviewers comment, without having read the paper, I would agree that just having predicates with a fixed interpretation is not really a major contribution. 
%% This is in a sense already present in the AIC formalism. Predicates that do not occur in rule heads are ``fixed''. Whether or not this fixed interpretation is shared over all databases is not a big deal? 
%% Anyway, as I said, I didn't read the unpublished work :p 
%% }


%% Modern databases are often equipped with integrity constraints that specify which are valid states of the database. 
%% Sometimes, databases can reach states in which one of the integrity constraints is violated, for instance due to execution of some invalid update, due to merging two databases, ... 
%% In such case, one needs to repair the database \cite{ai/EiterG92}. 
%% The problem with this is that it is often unclear which repairs should be considered ``good repairs''. 
%% This is where active integrity constraints (AICs)  kick in \mycite{AIC} . 
%% The idea of AICs is that instead of simply specifying which properties a valid state of the database should adhere to, one also specifies how violation of these properties can be repaired. 
%% However, when there are multiple active integrity constraints present, there may be dependencies between them. 
%% For instance, certain ways of fixing one AIC might result in a state where another AIC is violated. 
%%
%% lcf: not sure what to include from this list of references
%% Defining a natural semantics for such AICs has been a topic of intensive research \cite{iclp/CaropreseGSZ06,sebd/MolinaroGC06,iclp/CaropreseTZ07,dasfaa/CaropreseGM07,iclp/CaropreseT08,tkde/CaropreseGZ09,tplp/CaropreseT11,iclp/Cruz-Filipe16}. 

It is striking that many intuitions about what ``good'' repairs are, such as minimality of change, are similar to intuitions that surfaced in other domains of non-monotonic reasoning, such as logic programming \mycite{LP} and default logic \mycite{DL}. 
% For instance \todo{minimality} of change \cite{}, ... 
Still, it has been hard to find satisfying semantics for AICs.
%\todo{list some problems.}
As shown by~\citet{tase/Cruz-FilipeGEN13}, the semantics of so-called \emph{founded repairs}~\cite{iclp/CaropreseGSZ06} unexpectedly fails to respect the common-sense law of inertia, while the more restricted semantics of justified repairs~\cite{tplp/CaropreseT11} forbids natural repairs in some cases.
That work proposed the operational semantics of well-founded repairs, which however is not modular~\cite{foiks/Cruz-Filipe14} and is therefore severely restricted in its practical applicability.

% Essentially, fixing similar problems as in other NMR domains... 

In this work, we begin by defining a new semantics for AICs that avoids these problems: \emph{grounded repairs}.
Grounded repairs are natural counterparts to existing semantics in various non-monotonic reasoning domains such as logic programming; we discuss how they relate to other semantics for AICs.
We also argue that grounded repairs match our intuitions regarding AICs on a broad set of examples. 

We then give a more abstract characterization of the different semantics for AICs by associating with each set of AICs $\eta$ a semantic operator $\Op$.
This operator immediately induces several semantics:
\newcounter{hereiam}
\begin{enumerate}
 \item \emph{weak repairs}  are \emph{fixpoints} of $\Op$; 
 \item \emph{repairs} are \emph{minimal fixpoints} of $\Op$; 
 \item \emph{grounded repairs} are \emph{grounded fixpoints} \mycite{GroundedFixpoints} of $\Op$. 
 \setcounter{hereiam}{\value{enumi}}
\end{enumerate}
The first two semantics are pre-existing semantics for AICs that we recover in an operator-based fashion.

Next, we define a three-valued variant of $\Op$. In the terminology of \emph{approximation fixpoint theory (AFT)} \cite{\refto{AFT}} our three-valued operator is an \emph{approximator} of the original semantic operator.
Given such an approximator $\Ap$, AFT induces a few more semantics: 
\begin{enumerate}    \setcounter{enumi}{\value{hereiam}}
\item the \emph{Kripke-Kleene repair} is the \emph{Kripke-Kleene fixpoint} of $\Ap$; 
\item the \emph{AFT-well-founded repair} is the \emph{well-founded fixpoint} of $\Ap$; 
\item  \emph{(partial) stable repairs} are \emph{(partial) stable fixpoints} of $\Ap$;
% \item a Kripke-Kleene fixpoint semantics, and 
\item  \emph{partial grounded repairs} are \emph{partial grounded fixpoints} of $\Ap$. % semantics. %\bart{drop? non-classical} 
\end{enumerate} 
We again study properties of these new semantics and study how they compare to existing semantics. 
Furthermore, we argue that, from a practical point of view, the AFT-style well-founded semantics is very valuable. 
Indeed, we show that the AFT-well-founded repair can be computed in polynomial time, and that, on a broad set of practical examples, it corresponds to the intuitions underlying database repairs,  providing natural upper and lower bounds on the set of acceptable repairs (formally: the AFT-style well-founded model approximates all justified, stable and grounded repairs).

All our semantics are defined within the framework of approximation fixpoint theory, a general algebraic framework for studying logics with a fixpoint semantics.
This framework was initially developed by Denecker, Marek and Truszczy\'nski, henceforth referred to as DMT (\citeyear{DeneckerMT00}), after identifying analogies in the semantics of logic programming \mycite{LP}, autoepistemic logic (AEL) \cite{mo85} and default logic, hereafter abbreviated to DL \cite{ai/Reiter80}.
The theory defines different types of fixpoints for what are called \emph{approximating} operators, or \emph{approximators}.
In the context of logic programming, \citetDMT{DeneckerMT00} showed that Fitting's (three- or four-valued) immediate consequence operator is an approximator of the usual (two-valued) immediate consequence operator, and that the major semantics of logic programs coincide with the (equally named) different types of fixpoints of that approximator.
They then identified approximators for both default and autoepistemic logic, showing that AFT induces all main semantics in these fields, as well as some new ones \cite{DeneckerMT03}, thus unifying DL and AEL in a deep sense.
%They showed that Konolige's mapping from DL to AEL \cite{Konolige88} preserves the approximating operator and hence, preserves all of the types of semantics.  
%Thus, DL can (and should) be viewed as a fragment of AEL under Konolige's embedding and an old research question was resolved \cite{nonmon30/DeneckerMT11}. 
More recently, \citet{journals/ai/Strass13} showed that AFT can also be used to characterize the major semantics of Dung's argumentation frameworks \mycite{AF} and abstract dialectical frameworks \mycite{ADF}.
Other recent applications of AFT include: defining extensions of logic programming \cite{lpnmr/AnticEF13}, defining new logics \cite{iclp/BogaertsVDV14}, integrating different formalisms \cite{RR/BiJF14}, studying complexity \cite{kr/StrassW14}, and studying modularity and predicate introduction in a uniform way \cite{tocl/VennekensGD06,VennekensMWD07a,VennekensMWD07b}. 

%% Approximation fixpoint theory, the framework we use to define our semantics, was first defined by Denecker, Marek and Truszczy\'nski, henceforth referred to as DMT (\citeyear{DeneckerMT00}).
%% It is a general algebraical study of logics with a fixpoint semantics. 
%% \citetDMT{DeneckerMT00} developed this theory after discovering analogies in the semantics of logic programming \mycite{LP}, AEL \cite{mo85} and default logic, which we will abbreviate to DL \cite{ai/Reiter80}.
%% Their theory defines different types of fixpoints for a so-called approximating operator.
%% In the context of logic programming, they showed that Fitting's (three- or four-valued) immediate consequence operator is an approximator of the two-valued immediate consequence operator and that its different types of fixpoints correspond exactly with the  major, equally named, semantics of logic programs. 
%% They identified  approximating operators for default logic and autoepistemic logic and showed that AFT induces all main and some new semantics in these fields \cite{DeneckerMT03}. 
%% Moreover, this work unified DL and AEL in a deep sense.
%% They showed that Konolige's mapping from DL to AEL \cite{Konolige88} preserves the approximating operator and hence, preserves all of the types of semantics.  
%% Thus, DL can (and should) be viewed as a fragment of AEL under Konolige's embedding and an old research question was resolved \cite{nonmon30/DeneckerMT11}. 
%% Recently, \citet{journals/ai/Strass13} showed that also the major semantics of Dung's argumentation frameworks \mycite{AF} and abstract dialectical frameworks \mycite{ADF} can be characterized using AFT. 
%% Nowadays, AFT is used for various purposes, including defining extensions of logic programming \cite{lpnmr/AnticEF13}, defining new logics \cite{iclp/BogaertsVDV14}, integrating different formalisms \cite{RR/BiJF14}, studying complexity \cite{kr/StrassW14}, and studying modularity and predicate introduction in a uniform way \cite{tocl/VennekensGD06,VennekensMWD07a,VennekensMWD07b}. 

As such, the contribution of this work goes beyond the definition of new semantics for AICs. 
By integrating active integrity constraints in AFT, we provide solid foundations for applying a rich algebraic theory to AICs.
For instance, we can now directly apply existing results from AFT, such as modularity results and predicate introduction results to AICs. 
It remains to be researched how these related for instance to existing modularity results for AICs \cite{foiks/Cruz-Filipe14,iclp/Cruz-Filipe16}. 
% Similarly, AFT provides us with predicate introduction results \cite{VennekensMWD07a,VennekensMWD07b}. 
Furthermore, our work paves the way to applying AFT to revision programming, following the results from \citet{tplp/CaropreseT11}, and to AICs outside the database world, as generalized in \citet{ekaw/Cruz-Filipe16}.


% Besides defining a new class of interesting semantics, our work provides solid foundations for transferring a rich body of work from the non-monotonic reasoning domain to AICs. For instance, we can now directly apply existing results from AFT, such as modularity results \cite{tocl/VennekensGD06,tocl/BogaertsVD16} or predicate introduction results \cite{VennekensMWD07a,VennekensMWD07b}. 
% Following \citet{tplp/CaropreseT11}, our work also paves the way to applying AFT to revision programming. 

% \luis{I once refereed a paper that included paragraphs with titles ``contribution'' and ``publication history'', and thought it was a really good idea, so I now usually include them in journal submissions.} BART: OK
% \bart{I would drop the contributions paragarph but keep the history paragraph. 
% I find that the header ``contribution'' breaks the flow of the story a little bit. and everything that``s in this paragraph is now also in the one above}
% \paragraph{Contributions}
% The main contribution of this work is a family of new semantics for AICs, based on a semantic operator for AICs, inspired by the immediate consequence operator for logic programs, and its three valued variant (Fitting's operator).
% We argue that these semantics more intuitively capture the intended semantics of AICs, and include results on the complexity of computing them.
% The results from this paper also shed a new light on the connection between active integrity constraints and logic programming. 
% 
% 
% Besides defining a new class of interesting semantics, our work provides solid foundations for applying a rich algebraic theory to AICs. For instance, we can now directly apply existing results from AFT, such as modularity results \cite{tocl/VennekensGD06,tocl/BogaertsVD16}. It remains to be researched how these algebraic modularity results relate to existing modularity results for AICs \cite{foiks/Cruz-Filipe14,iclp/Cruz-Filipe16}. 
% Similarly, AFT provides us with predicate introduction results \cite{VennekensMWD07a,VennekensMWD07b}. 
% In view of the results from \citet{tplp/CaropreseT11}, our work also paves the way to applying AFT to revision programming.

The rest of this paper is structured as follows. In Section \ref{sec:prelims}, we provide preliminaries related to active integrity constraints. 
In Section \ref{sec:grounded} we discuss the semantics of grounded repairs. While our definitions are motivated from approximation fixpoint theory, their semantics can also be given without this machinery, hence we start with a direct definition. 
Next, in Section \ref{sec:aft}, we give background on approximation fixpoint theory. 
In Section \ref{sec:operator}, we define a semantic operator for AICs and show that its grounded fixpoints indeed correspond to grounded repairs, as defined in Section \ref{sec:grounded}.
Next, we define an approximator of our operator in Section \ref{sec:approximator} and use it to derive more AFT-style semantics for AICs; we study how these semantics relate to existing semantics. 
Afterwards, in Section \ref{sec:lp}, we discuss the relationship between our newly defined semantics and the equally-named semantics for logic programming. 
In Section \ref{sec:complexity}, we study complexity of various tasks related to our newly defined semantics. We conclude in Section \ref{sec:concl}.


\paragraph{Publication history}
The semantic operator for grounded AICs and the resulting semantics of grounded repairs were originally proposed by \citet{iclp/Cruz-Filipe16}.
The approximator for this operator and its properties were introduced by \citet{ijcai/BogaertsC17}.
Our current work combines results from those conference papers and extends it with proofs, examples, and a detailed analysis of the connection between the approximation semantics for AICs and logic programming.
% \luis{anything else?}
% \bart{I also added examples}

% To establish this relationship, we define a family of new semantics for AICs with some desirable properties:
% \begin{compactitem}
%  \item Each of our semantics satisfies the so-called \emph{shifting property} \cite{tcs/MarekT98}. 
%  \item The AFT-based well-founded semantics provides good over and under approximations of various other semantics, resulting in a computationally attractive method for reasoning approximately in one of these semantics. 
%  \item The AFT-based stable semantics is closely related to the grounded fixpoint semantics, but computationally more attractive.
% \end{compactitem}

% 
% Somewhere we should summarize our contributions:
% \begin{compactitem}
%  \item Our work provides better insights in the relationship between logic programming (or AEL, DL, ADF, AF) and AICs
% \item We define a new set of semantics with desirable properties:
% \begin{compactitem}
% \item  Shifting
%  \item WFF: good over and under approximation of what repairs should look like
%  \item WFF: allows us already to answer queries over the repaired database without computing (or deciding on) an actual repair. (not every query can be answered of course). 
%  \item WFF: polytime
%  \item SS: very close to grounded fixpoints, but computationally more attractive
% \end{compactitem}
% \item Our work opens the way to applying a rich algebraic theory to AICs: stratification and modularity results \cite{tocl/VennekensGD06,tocl/BogaertsVD16} and predicate introduction results \cite{VennekensMWD07a,VennekensMWD07b}. 
% \item Following results from \citet{tplp/CaropreseT11}, our work also opens the way to applying AFT to revision programming. 
% \end{compactitem}
%  


%%% Local Variables: 
%%% mode: latex
%%% TeX-master: "../AFT-semantics-AIC"
%%% End: 
