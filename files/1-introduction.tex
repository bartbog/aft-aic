One of the key components of modern-day databases are integrity constraints: logical formulas that specify semantic relationships between the data being modeled that have to be satisfied at all times.
When the database is changed (typically by updating), it is necessary to check if its integrity constraints still hold; in the negative case, the database must be repaired.

The problem of database repair has been an important topic of research for more than thirty years~\cite{icdt/Abiteboul88}.
There are two major problems when deciding how to repair an inconsistent database: \emph{finding} possible repairs and \emph{choosing} which one to apply.
Indeed, there are typically several ways to fix an inconsistent database, and several criteria to choose the ``best'' one have been proposed over the years.
Among the most widely accepted criteria are minimality of change~\cite{Winslett90,ai/EiterG92} -- change as little as possible -- and the common-sense law of inertia~\cite[discussed in, e.g.,][]{PrzymusinskiT97} -- do not change anything \longpaper{unless there is a reason for the change.}\shortpaper{without a reason to do it.}

A typical implementation of integrity constraints in database systems is by means of event-condition-action rules~\cite{TenienteO95,WidomC96}, which specify update actions to be performed when a particular event (a trigger) occurs and specific conditions hold.
However, these rules do not have a declarative semantics, making their joint behavior hard to understand.
\luis{add more detail here}

The formalism of \emph{active integrity constraints (AICs)} \cite{ppdp/FlescaGZ04} was inspired by a similar idea.
AICs express database dependencies through logic programming-style rules that include update actions in their heads.
They come with a set of declarative semantics~\cite{tplp/CaropreseT11} that identifies several progressively more restricted classes of repairs, which can be used as criteria to select a preferred repair.
These repairs can be computed \longpaper{directly }by means of tree algorithms\longpaper{~\cite{tase/Cruz-FilipeEGN13}, which have been implemented as a prototype~\cite{CFHLNS15}.}\shortpaper{\cite{tase/Cruz-FilipeEGN13,CFHLNS15}.}

%% Modern databases are often equipped with integrity constraints that specify which are valid states of the database. 
%% Sometimes, databases can reach states in which one of the integrity constraints is violated, for instance due to execution of some invalid update, due to merging two databases, ... 
%% In such case, one needs to repair the database \cite{ai/EiterG92}. 
%% The problem with this is that it is often unclear which repairs should be considered ``good repairs''. 
%% This is where active integrity constraints (AICs)  kick in \mycite{AIC} . 
%% The idea of AICs is that instead of simply specifying which properties a valid state of the database should adhere to, one also specifies how violation of these properties can be repaired. 
%% However, when there are multiple active integrity constraints present, there may be dependencies between them. 
%% For instance, certain ways of fixing one AIC might result in a state where another AIC is violated. 
%%
%% lcf: not sure what to include from this list of references
%% Defining a natural semantics for such AICs has been a topic of intensive research \cite{iclp/CaropreseGSZ06,sebd/MolinaroGC06,iclp/CaropreseTZ07,dasfaa/CaropreseGM07,iclp/CaropreseT08,tkde/CaropreseGZ09,tplp/CaropreseT11,iclp/Cruz-Filipe16}. 

It is striking that many intuitions about \longpaper{what }``good'' repairs\longpaper{ are}, such as minimality of change, are similar to intuitions that surfaced in other domains of non-monotonic reasoning, such as logic programming \mycite{LP} and default logic \mycite{DL}. 
% For instance \todo{minimality} of change \cite{}, ... 
Still, it has been hard to find satisfying semantics for AICs.
%\todo{list some problems.}
As shown by~\citet{tase/Cruz-FilipeEGN13}, the semantics of so-called \emph{founded repairs}~\cite{iclp/CaropreseGSZ06} unexpectedly fails to respect the common-sense law of inertia, while the more restricted semantics of justified repairs~\cite{tplp/CaropreseT11} forbids natural repairs\longpaper{ in some cases}.
That work proposed the operational semantics of well-founded repairs, which however is not modular~\cite{foiks/Cruz-Filipe14} and therefore \longpaper{severely }restricted in its practical applicability.

% Essentially, fixing similar problems as in other NMR domains... 


In this work, we define a new class of semantics for AICs that are natural counterparts of existing semantics in various non-monotonic reasoning domains. 
To achieve this, we associate with each set of AICs a semantic operator. This operator immediately induces a couple of semantics:
\newcounter{hereiam}
\begin{enumerate}
 \item \emph{weak repairs}  are \emph{fixpoints} of the operator; 
 \item \emph{repairs} are \emph{minimal fixpoints} of the operator; 
 \item \emph{grounded repairs} are \emph{grounded fixpoints} \mycite{GroundedFixpoints} of the operator. 
 \setcounter{hereiam}{\value{enumi}}
\end{enumerate}
The former two are existing semantics for AICs that are recovered here in an operator-based fashion. The latter is a new semantics. We study how this semantics relates to existing semantics and argue that it corresponds with intuitions regarding AICs on a broad set of examples. 

Next, we define a three-valued variant of our operator. In the terminology of \emph{approximation fixpoint theory (AFT)} \cite{\refto{AFT}} our three-valued operator is an \emph{approximator} of the original semantic operator. 
Given such an approximator, AFT induces a couple more semantics: 
\begin{enumerate}    \setcounter{enumi}{\value{hereiam}}
\item the \emph{Kripke-Kleene repair} is the \emph{Kripke-Kleene fixpoint} of the approximator; 
\item the \emph{AFT-well-founded repair} is the \emph{well-founded fixpoint} of the approximator; 
\item  \emph{(partial) stable repairs} are \emph{(partial) stable fixpoints} of the operator;
% \item a Kripke-Kleene fixpoint semantics, and 
\item  \emph{partial grounded repairs} are \emph{partial grounded fixpoints} of the operator. % semantics. %\bart{drop? non-classical} 
\end{enumerate} 
We study properties of these new semantics and study how they compare to existing semantics. 
Furthermore, we argue that, from a practical point of view, the AFT-style well-founded semantics is very valuable. 
Indeed, we show that this semantics can be computed in polynomial time, and that, on a broad set of practical examples, it corresponds to the intuitions underlying database repairs,  providing natural upper and lower bounds on the set of acceptable repairs (formally: the AFT-style well-founded model approximates all justified, stable and grounded repairs).

Approximation fixpoint theory, the framework we use to define our semantics, was first defined by Denecker, Marek and Truszczy\'nski (henceforth referred to as DMT) (\citeyear{DeneckerMT00}). 
It is a general algebraical study of logics with a fixpoint semantics. 
\citetDMT{DeneckerMT00} developed this theory after discovering analogies in the semantics of logic programming \mycite{LP}, AEL \cite{mo85} and  default logic (DL) \cite{ai/Reiter80}.
Their theory defines different types of fixpoints for a so-called approximating operator.
In the context of logic programming, they showed that Fitting's (three- or four-valued) immediate consequence operator is an approximator of the two-valued immediate consequence operator and that its different types of fixpoints correspond exactly with the  major, equally named, semantics of logic programs. 
They identified  approximating operators for default logic and autoepistemic logic and showed that AFT induces all main and some new semantics in these fields \cite{DeneckerMT03}. 
Moreover, this work unified DL and AEL in a deep sense.
They showed that Konolige's mapping from DL to AEL \cite{Konolige88} preserves the approximating operator and hence, preserves all of the types of semantics.  
Thus, DL can (and should) be viewed as a fragment of AEL under Konolige's embedding and an old research question was resolved \cite{nonmon30/DeneckerMT11}. 
Recently, \citet{journals/ai/Strass13} showed that also the major semantics of Dung's argumentation frameworks \mycite{AF} and abstract dialectical frameworks \mycite{ADF} can be characterized using AFT. 
Nowadays, AFT is used for various purposes.  These purposes include defining extensions of logic programming \cite{lpnmr/AnticEF13}, defining new logics \cite{iclp/BogaertsVDV14}, integrating different formalisms \cite{RR/BiJF14}, studying complexity \cite{kr/StrassW14}, and studying modularity and predicate introduction for logics with a fixpoint semantics in a uniform way \cite{tocl/VennekensGD06,VennekensMWD07a,VennekensMWD07b}. 

As such, the merits of this paper are not just that we define new semantics for AICs. 
By integrating active integrity constraints in AFT, we provide  solid foundations for applying a rich algebraic theory to AICs.
For instance, we can now directly apply existing results from AFT, such as modularity results \cite{tocl/VennekensGD06,tocl/BogaertsVD16}. It remains to be researched how these algebraic modularity results relate to existing modularity results for AICs \cite{foiks/Cruz-Filipe14,iclp/Cruz-Filipe16}. 
Similarly, AFT provides us with predicate introduction results \cite{VennekensMWD07a,VennekensMWD07b}. 
In view of the results from \citet{tplp/CaropreseT11}, our work also opens the way to applying AFT to revision programming.




% Besides defining a new class of interesting semantics, our work provides solid foundations for transferring a rich body of work from the non-monotonic reasoning domain to AICs. For instance, we can now directly apply existing results from AFT, such as modularity results \cite{tocl/VennekensGD06,tocl/BogaertsVD16} or predicate introduction results \cite{VennekensMWD07a,VennekensMWD07b}. 
% Following \citet{tplp/CaropreseT11}, our work also paves the way to applying AFT to revision programming. 

% \luis{I once refereed a paper that included paragraphs with titles ``contribution'' and ``publication history'', and thought it was a really good idea, so I now usually include them in journal submissions.} BART: OK
\bart{I would drop the contributions paragarph but keep the history paragraph. 
I find that the header ``contribution'' breaks the flow of the story a little bit. and everything that``s in this paragraph is now also in the one above}
\paragraph{Contributions}
The main contribution of this work is a family of new semantics for AICs, based on a semantic operator for AICs, inspired by the immediate consequence operator for logic programs, and its three valued variant (Fitting's operator).
We argue that these semantics more intuitively capture the intended semantics of AICs, and include results on the complexity of computing them.
The results from this paper also shed a new light on the connection between active integrity constraints and logic programming. 


Besides defining a new class of interesting semantics, our work provides solid foundations for applying a rich algebraic theory to AICs. For instance, we can now directly apply existing results from AFT, such as modularity results \cite{tocl/VennekensGD06,tocl/BogaertsVD16}. It remains to be researched how these algebraic modularity results relate to existing modularity results for AICs \cite{foiks/Cruz-Filipe14,iclp/Cruz-Filipe16}. 
Similarly, AFT provides us with predicate introduction results \cite{VennekensMWD07a,VennekensMWD07b}. 
In view of the results from \citet{tplp/CaropreseT11}, our work also paves the way to applying AFT to revision programming.



\paragraph{Publication history}
The semantic operator for grounded AICs and the resulting semantics of grounded repairs were originally proposed by \citet{iclp/Cruz-Filipe16}.
The approximator for this operator and its properties were introduced by \citet{ijcai/BogaertsC17}.
Our current work combines results from those conference papers and extends it with proofs and a detailed analysis of the connection between the approximation semantics for AICs and logic programming.

% To establish this relationship, we define a family of new semantics for AICs with some desirable properties:
% \begin{compactitem}
%  \item Each of our semantics satisfies the so-called \emph{shifting property} \cite{tcs/MarekT98}. 
%  \item The AFT-based well-founded semantics provides good over and under approximations of various other semantics, resulting in a computationally attractive method for reasoning approximately in one of these semantics. 
%  \item The AFT-based stable semantics is closely related to the grounded fixpoint semantics, but computationally more attractive.
% \end{compactitem}

% 
% Somewhere we should summarize our contributions:
% \begin{compactitem}
%  \item Our work provides better insights in the relationship between logic programming (or AEL, DL, ADF, AF) and AICs
% \item We define a new set of semantics with desirable properties:
% \begin{compactitem}
% \item  Shifting
%  \item WFF: good over and under approximation of what repairs should look like
%  \item WFF: allows us already to answer queries over the repaired database without computing (or deciding on) an actual repair. (not every query can be answered of course). 
%  \item WFF: polytime
%  \item SS: very close to grounded fixpoints, but computationally more attractive
% \end{compactitem}
% \item Our work opens the way to applying a rich algebraic theory to AICs: stratification and modularity results \cite{tocl/VennekensGD06,tocl/BogaertsVD16} and predicate introduction results \cite{VennekensMWD07a,VennekensMWD07b}. 
% \item Following results from \citet{tplp/CaropreseT11}, our work also opens the way to applying AFT to revision programming. 
% \end{compactitem}
%  


%\luis{Despite all the information in the background, I guess one can only understand this if one already knows something about AFT. We should state it explicitly in the beginning.}
%\bart{You mean in the beginning of the example or of the paper?}
%\luis{I guess in the beginning of the paper... just prepare people psychologically so that they won't be scared off too easily :-)}

%%% Local Variables: 
%%% mode: latex
%%% TeX-master: "../AFT-semantics-AIC"
%%% End: 
