\subsection{Preliminaries: Logic Programming (and AFT)}
\renewcommand\voc\atoms
% Let \voc be an alphabet, i.e., a collection of symbols which are called \emph{atoms}. A \emph{literal} is an atom $p$ or the negation $\lnot q$ of an atom $q$.  
A logic program $\mathcal{P}$ is a set of \emph{rules} $r$ of the form 
$h\lrule \varphi$, where
$h$ is an atom called the \emph{head} of $r$, denoted $head(r)$, and $\varphi$ is a propositional formula called  the \emph{body} of $r$, denoted $body(r)$.
% An interpretation $\struct$ of the alphabet \voc is an element of $2^\voc$, i.e., a subset of $\voc$.  
The set of interpretations $2^\voc$ forms a lattice equipped with the order $\subseteq$. 
The truth value (\ltrue or \lfalse) of a propositional formula $\varphi$ in a structure $\struct$, denoted $\varphi^\struct$ is defined as usual.
With a logic program \PP, we associate an immediate consequence operator \cite{jacm/EmdenK76} $T_\PP$ that maps a structure $\struct$ to 
	\[T_\PP(\struct) = \{p\mid \exists r \in \PP: head(r)=p\land body(r)^\struct=\ltrue\}.\] 
	
	
	In the context of logic programming, elements of the bilattice $\left(2^\voc\right)^2$ are four-valued interpretations, pairs $\pstruct= (\struct_1,\struct_2)$ of interpretations.
The pair $(\struct_1,\struct_2)$ approximates all interpretations $\struct'$ with $\struct_1\subseteq \struct'\subseteq \struct_2$.
We often identify an interpretation $I$ with the four-valued interpretation $(I,I)$.
% If $\pstruct=(I_1,I_2)$ is a (four-valued) interpretation, and $U\subseteq \voc$, we write $\pstruct[U:\lfalse]$ for the (four-valued) interpretation that equals $\pstruct$ on all elements not in $U$ and that interprets all elements in $U$ as $\lfalse$, i.e., the interpretation $(I_1\setminus U, I_2\setminus U)$. 
We are mostly concerned with consistent (also called partial or three-valued) interpretations: tuples $\pstruct=(\struct_1,\struct_2)$ with $\struct_1\subseteq\struct_2$. 
For such an interpretation, the atoms in $\struct_1$ are \emph{true} (\ltrue) in $\pstruct$, the atoms in $\struct_2\setminus\struct_1$ are \emph{unknown} (\lunkn) in $\pstruct$ and the other atoms are \emph{false} (\lfalse) in $\pstruct$. 
If $\pstruct$ is a three-valued interpretation, and $\varphi$ a formula, we write $\varphi^\pstruct$ for the standard three-valued valuation based on the Kleene truth tables (see Figure \ref{fig:KT}). 




Several approximators have been defined for logic programs. The most common is Fitting's immediate consequence operator $\Psi_\PP$ \cite{tcs/Fitting02}, a direct
generalization of $T_\PP$ to partial interpretations:
  \begin{align*}
    \Psi_\PP(\pstruct)_1 &=\{a\in \voc\mid body(r)^\pstruct=\ltrue \text{ for some rule $r\in \PP$ with }head(r)=a\},\\
    \Psi_\PP(\pstruct)_2 &=\{a\in \voc\mid body(r)^\pstruct\neq \lfalse \text{ for some rule $r\in \PP$ with }head(r)=a\}.
   \end{align*}
\citet{DeneckerMT00} showed that $\Psi_\PP$ is an approximator of $T_\PP$, that  the well-founded fixpoint of $\Psi_\PP$ is the well-founded model of $\PP$ as defined by \citeauthor{GelderRS91} and that $\Psi_\PP$-stable fixpoints are exactly the stable models of $\PP$ as defined by \citeauthor{iclp/GelfondL88}. In this case, 
the operator $\Psi_\PP(\cdot,y)_1$ coincides with the immediate consequence operator of the Gelfond-Lifschitz reduct \cite{iclp/GelfondL88}. 




\subsection{AIC and Logic Programs}
\citet{tplp/CaropreseT11} defined a translation from logic programs to AICs as follows. 
\begin{definition}
 Let $r$ be a normal logic programming rule, 
 \[a\leftarrow \lit_1,\dots,\lit_n.\]
 We define the active integrity constraint $\aicop r$ as 
 \[\lit_1\wedge\ldots\wedge\lit_n\wedge\neg a \aicrule \add a.\]
 Furthermore, if \PP is a normal logic program, we define 
 \[\aicop(\PP)=\bigcup\{\aicop(r)\mid r\in \PP\}.\]
\end{definition}

\citet{tplp/CaropreseT11} showed that an interpretation $I$ is a stable model of \PP if and only if $I$ (viewed as an update set) is a justified repair of $\aicop(\PP)$. Since $\aicop(\PP)$ is unipolar, from our earlier results (Propositions \ref{prop:stable_is_justified} and \ref{prop:justified_is_stable_sometimes}) it follows that this is also equivalent with the condition that $I$ is a $\threeap_{\aicop(\PP)}$-stable repair. 
The same result is also a corollary of the following stronger theorem. 
\begin{theorem}\label{thm:partialstable-LP}
 Let $\PP$ be a normal logic program and $\pstruct$ a partial interpretation. It holds that $\pstruct$ is a partial stable model of \PP if and only if $\pstruct$ is a partial stable repair of $\langle \aicop(\PP),\emptyset\rangle$. 
\end{theorem}
The proof of this theorem follows later, since it makes use of proposition \ref{prop:lplink}. 

\begin{corollary}
 Let $\PP$ be a normal logic program. The well-founded model of \PP coincides with the AFT-well-founded repair of $\langle \aicop(\PP),\emptyset\rangle$. 
\end{corollary}

 While the operation $\aicop$ preserves (partial) stable, well-founded  fixpoints, it does not preserve grounded fixpoints or the Kripke-Kleene fixpoint (as the following two examples illustrate). In both cases, it is the intuition of \emph{inertia}, present in AICs but not in logic programs, that is responsible for the difference. 

 
\begin{example}\label{ex:grounded}
%   While the operation $\aicop$ preserves various types of AFT fixpoints, it does not preserve grounded fixpoint. 
  Consider the logic program 
  \[\PP_g = \left\{p\lrule \lnot p \right\}.\]
  This program has no fixpoints, hence definitely no grounded fixpoints. 
  In this case, 
  \[\aicop(\PP_g) = \left\{\lnot p \aicrule \add p\right\}.\]
  Now, $\langle \aicop(\PP_g),\emptyset\rangle$ has one grounded repair, namely $\{\add p\}$. 
\end{example}
\bart{WAIT!!!! I'm getting confused here. 
In this example, isn't the AFT-well-founded repair $\{+p\}$, which is the nalso a stable repair? That would contradict Theorem \ref{thm:partialstable-LP} and also the combination of Propositions \ref{prop:stable_is_justified} and \ref{prop:justified_is_stable_sometimes}!

7.2 surely is wrong. THe proof is simply bogus. $lp\circ aic \neq triv$ in general. Only if the logic program does not have the negation of its defined literal in the body. 

Also: the claim of Caroprese and Truzczynski ONLY holds for SIMPLE logic programs (Pg is NOT simple) . To be fixed!!!! 

}

\begin{example}
 Consider the logic program
 \[\PP_{kk}=\left\{ b\lrule a\right\}.\]
 The Kripke-Kleene model of this program maps both $a$ and $b$ to $\lfalse$. 
 The corresponding AIC 
 \[\aicop(\PP_{kk}) = \{\lnot b \land a \aicrule \add b\}\] 
 has a Kripke-Kleene repair that maps $a$ and $b$ to $\lunkn$, i.e., it is unknown if these need to be changed. 
\end{example}


It can be seen from the previous example that for AICs, the Kripke-Kleene semantics is very bad at deriving that something does \emph{not} need to be changed. The well-founded semantics is much stronger with that respect. 
In a certain sense, one might say that the intuition of ``inertia'' underlying AICs lies at the foundation of this discrepancy. 
Now, in logic programming, the Kripke-Kleene semantics exhibits similar behavior. Consider for instance the empty logic program 
\[ \PP_\emptyset = \{ \}.\]
This program has a Kripke-Kleene model in which each atom is false, as is to be expect. 
However, adding a trivial rule
\[p\lrule p\]
 that ``simulates'' inertia results in a program with a Kripke-Kleene model in which $p$ is unknown.
 
 Inertia is also responsible for the discrepancy in Example \ref{ex:grounded}. 
 Intuitively, $\aicop(\PP_g)$ from that example corresponds more to the logic program
  \[\PP_g' = \left\{\begin{array}{l}p\lrule \lnot p \\ p\lrule p  \end{array}\right\}.\]
  Indeed $\aicop(\PP_g)$ states that if $p$ is not present in the database, add it, and once we add it, by inertia, it stays unless there is a reason to remove it again (which there is not). 
  $\PP'$ and $\PP_g$ only differ from each other in the rule $p\lrule p$, which simulates inertia.
 
 
 The above discussion provides intuitions on what a transformation that preserves all AFT semantics should look like. In the following proposition, this is formalized. 

  


\begin{proposition}\label{prop:lplink}
  Suppose that $\db=\emptyset$ and that the only update actions in \aics are of the form $\add a$.
  Let $\lpop(\aics)$ denote the following logic program:  
 \begin{align*}
  \lpop(\aics)=& \{a \lrule \nup(r)\mid r\in \aics, \head(r)= \add a\}\\
  &\cup \{a \lrule a\mid a \in \atoms\}
 \end{align*}
 then $\Ap = \Psi_{\lpop(\aics)}$. 

Hence, all AFT semantics for \fulldb coincide in this case with the equally-named semantics for the logic program $\lpop(\aics)$. 
\end{proposition}
\begin{proof}
  \luis{Proof has been done by hand. To be added.}
  By case analysis on the definition of \Ap.
  The condition on \fulldb guarantees that $\suppout$ is always false.
  The inclusion of the rules of the form $a\lrule a$ is necessary to mimic the common-sense law of inertia, which is embodied in the semantics for AICs but not in that of logic programs.
\end{proof}



\begin{proof}[Proof of Theorem \ref{thm:partialstable-LP}]
 For each logic program \PP, let $\triv(\PP)$ denote the logic program 
 \[
  \triv(\PP) = \PP \cup \{p\lrule p\mid p\in \atoms\}.
 \]
 It is well-known\footnote{For completeness, we give the argument. Let $I$ be an interpretation; from the definition of partial stable fixpoint, it suffices to show that 
 $\lfp(\Psi_\PP(\cdot,I)_1)= \lfp(\Psi_{\triv(\PP)}(\cdot,I)_1)$ and that 
  $\lfp(\Psi_\PP(I,\cdot)_2)= \lfp(\Psi_{\triv(\PP)}(I,\cdot)_2)$. Since we are working with symmetric operators, it suffices to prove the first of the two equalities. 
  Now, it is easy to see that $\Psi_{\triv(\PP)}(\cdot,I)_1$ is the inflationary operator of $\Psi_\PP(\cdot,I)_1$, i.e., that $\Psi_{\triv(\PP)}(\cdot,I)_1(J) = \Psi_\PP(\cdot,I)_1(J)\cup J$. Hence, these two monotone operators have the same prefixpoints and thus also the same least prefixpoint, which equals their least prefixpoint. 
  }
 that the partial stable models of $\PP$ and those of $\triv(\PP)$ are the same. 
 Now, the result easily follows from the facts that 
 \[\lpop(\aicop(\PP)) = \triv(\PP)\]
 and that $\triv$ and $\lpop$ preserve partial stable models. 
% 
\end{proof}


\begin{example}
 Proposition \ref{prop:lplink} does not hold in general, not even for unipolar AICs. 
 Consider for instance the following sets of AICs. 
 \begin{align*}\eta_1 &= \left\{\begin{array}{l}
                    a \aicrule \remove a\\
                     b  \aicrule \remove b
                   \end{array}\right\}\\
                   \eta_2 &=\emptyset.
\end{align*}
These two are not equivalent under all AFT semantics. 
For the first one, the KK-repair is $\{a\mapsto\lfalse,b\mapsto\lfalse\}$, while for the latter, the KK-repair is $\{a\mapsto\lunkn,b\mapsto\lunkn\}$. However, they have the same translation to logic programs (since there are no rules with positive actions in the head).          
%                    
% 
\end{example}

The reason why the equivalence does not hold in the previous example is because rules with $\remove a$ in the head are ignored, while in the context of AICs such rules can make a semantic difference. 

%%% Local Variables:
%%% mode: latex
%%% TeX-master: "../AFT-semantics-AIC.tex"
%%% End:
