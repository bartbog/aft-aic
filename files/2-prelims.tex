In this section we summarize previous work on active integrity constraints (AICs), including results developed by \citet{ppdp/FlescaGZ04,iclp/CaropreseGSZ06,tplp/CaropreseT11} and \citet{tase/Cruz-FilipeEGN13}.

We assume a fixed set \atoms of \emph{atoms}. An \emph{interpretation} or \emph{database} is a subset of \atoms. A \emph{literal} is an atom $a$ or its negation $\lnot a$. 
We say that $\lnot a$ is the \emph{dual} literal of $a$ and vice versa, and denote the dual of a literal $l$ by $l^D$. 
Propositional formulas are defined as usual: atoms are formulas, the negation of a formula is a formula, and the conjunction of formulas is a formula. 
The satisfaction relation between databases \db and literals is defined as usual: 
\begin{itemize}
 \item $\db \models a$ if $a\in\db$,
 \item $\db \models \lnot \varphi$ if $\db\not\models \varphi$,
% \item $\db \models \varphi\lor\psi$ if $\db\models \varphi$ or $\db \models \psi$,
 \item $\db \models \varphi\land\psi$ if $\db\models \varphi$ and $\db \models \psi$
\end{itemize}
for all atoms $a$ and all formulas $\varphi$ and $\psi$.
% \luis{commented out disjunction: do we need it?} NO

An \emph{update action} has the form $\add a$ or $\remove a$ with $a\in \atoms$. We call $\add a$ and $\remove a$ \emph{dual actions} and use $\alpha^D$ to denote the dual action of $\alpha$. 
Intuitively, update actions represent changes to the database: $\add a$ adds $a$, while $\remove a$ removes $a$. Formally, $\add a$ transforms $\db$ into $\db\cup\{a\}$, and $\remove a $ transforms $\db$ into $\db\setminus\{a\}$.
A set of update actions $\UU$ is \emph{consistent} if it does not contain both an action and its dual. A consistent set of update actions \UU acts on a database \db by executing all its actions simultaneously; we denote the result of this operation by $\UU(\db)$. 

Literals and update actions are related by mappings \litof and \actof, where $\litof(\add a)= a, \litof(\remove a) = \lnot a$ and $\actof$ is the inverse of $\litof$.
%, \actof(a) = \add a$ and $\actof(\lnot a) = \remove a$.  
These mappings naturally extend to sets of literals/actions. 

\begin{definition}
 An \emph{active integrity constraint} (AIC) is a rule $r$ of the form
%  \todo{replace $\mid$ by $\lor$?}
  \begin{equation} l_1\land \dots\land l_n\aicrule \alpha_1\mid\dots \mid \alpha_k \tag{1}\label{eq:aic}\end{equation}
  such that $\litof(\alpha_i^D) \in \{l_1,\dots,l_n\}$ for each $i$.
%  
  We call $l_1\land \dots \land l_n$ the \emph{body} of $r$, denoted $\body(r)$, and $\alpha_1\mid\dots \mid \alpha_k$ the head of $r$, denoted $\head (r)$.
\end{definition}

The informal reading of the above rule is: ``If each of the $l_i$ holds in \db, then \db is inconsistent. 
It is allowed to repair this inconsistency by executing execute one or more of the $\alpha_i$.''
An AIC represents a constraint a database should adhere to and, in addition, which atoms can be changed in order to fix it. Intuitively, atoms should only be changed if there is some rule that allows it.
Furthermore, the only actions that are able to repair the inconsistency detected by the body of an AIC are those corresponding to the duals of its literals~\cite{iclp/CaropreseGSZ06}, hence we restrict the actions allowed in the head to these.


An AIC is called \emph{normal} if $k=1$. The \emph{normalization} of an AIC of the form \eqref{eq:aic} is the set of AICs 
\[\{l_1\land \dots\land l_n\aicrule \alpha_i\mid 1\leq i\leq k\}.\]
It follows from the informal explanation above that we expect normalization to preserve semantics. Indeed, this is the case for most semantics of AICs -- the notorious exception being the semantics of justified repairs \cite{tplp/CaropreseT11}, which also poses several other problems~\cite{tase/Cruz-FilipeEGN13}.
In the current paper, we assume that, unless explicitly mentioned otherwise, \emph{all AICs are normal}. Extensions of the semantics we define for non-normal AICs can be obtained through normalization, if needed.

\begin{definition}
 A set of update actions \UU is a \emph{weak repair} for \db and a set \aics of AICs (shortly, for $\fulldb$) if:
 \begin{itemize}
  \item every action in \UU changes \db, and 
  \item $\UU(\db)\not\models \body(r)$ for each $r\in\aics$.
 \end{itemize}
A $\subseteq$-minimal weak repair is called a \emph{repair}.
\end{definition}

(Weak) repairs do not take the head of AICs into account, and thus allow arbitrary changes to the database.

\begin{example}
  \label{ex:repairs}
  Consider the restriction that ``if $a$ and $b$ both hold, then $c$ and $d$ should also hold''.
%   Since the syntax of AICs does not allow disjunction, we need two clauses to write this constraint.
  The following set of AICs expresses these constraints, only allowing repair actions of the form ``remove $a$'' or ``remove $b$''.
  \begin{align*}
    a\land b\land\neg c &\supset -a\mid -b\\    
    a\land b\land\neg d &\supset -a\mid -b
  \end{align*}
  Despite the inclusion of explicit repair actions in the heads of these rules, the notions of weak repair and repair do not take them into account.
  Suppose a given database is $\db=\{a,b\}$.
  Then both $\{{-a}\}$, $\{{-b}\}$ and $\{{+c},{+d}\}$ are repairs; furthermore, sets such as $\{{-a},{-b}\}$ or $\{{-b},{+c}\}$ are weak repairs: in all cases, all actions change $\db$, and the result always negates at least one literal in the body of each of the rules.

  Sets such as $\{{-a},{-c}\}$ and $\{{+a},{+c},{+d}\}$ also solve the inconsistency, but they include actions that do not change the database, and therefore are not considered weak repairs.
  Sets such as $\{{+a},{-a}\}$ are inconsistent, as it is not clear whether they state that the update of $\db$ should include $a$ or not.

  Applying normalization yields the following set of AICs.
  \begin{align*}
    a\land b\land\neg c &\supset{-a} \\
    a\land b\land\neg d &\supset{-a} \\
    a\land b\land\neg c &\supset{-b} \\
    a\land b\land\neg d &\supset{-b}
  \end{align*}
  It is immediate to check that everything discussed above with respect to the original set of AICs also applies to its normalized counterpart.
\end{example}

 We now review several other semantics for AICs that have been defined with the intention to allow only changes explicitly allowed by one of the AICs: founded (weak) repairs \cite{iclp/CaropreseGSZ06}, justified (weak) repairs \cite{tplp/CaropreseT11}, and well-founded (weak) repairs \cite{tase/Cruz-FilipeEGN13}.

\begin{definition}[\citenb{iclp/CaropreseGSZ06}]
 A set of update actions \UU is \emph{founded} with respect to $\fulldb$ if, for each $\alpha \in \UU$, there is a rule  $r\in \aics$ with $\alpha\in\head(r)$ and such that $\UU'(\db)\models \body(r)$, where $\UU'=\UU\setminus\{\alpha\}$. A \emph{founded (weak) repair} is a (weak) repair that is founded.
\end{definition}

The intuition behind this definition is that for each element in a ``good'' repair there should be a reason such that: if the element is removed, some constraint is violated and the removed element is an allowed fix. 


\begin{example}
  Consider again the database $\db=\emptyset$ together with the set $\eta$ of normalized AICs from the previous example.
  \begin{align}
    a\land b\land\neg c &\supset{-a} \label{rfok1} \\
    a\land b\land\neg d &\supset{-a} \label{rfok2} \\
    a\land b\land\neg c &\supset{-b} \label{rfok3} \\
    a\land b\land\neg d &\supset{-b} \label{rfok4}
  \end{align}

  The set $\{{-a}\}$ is founded with respect to $\fulldb$: if its only action is removed, then rule \ref{rfok1} is applicable, and ${-a}$ occurs in $\head(\ref{rfok1})$.
  Likewise, the set $\{{-b}\}$ is also a founded repair for $\fulldb$.

  On the other hand $\UU=\{{+c},{+d}\}$ is not founded.
  If we remove e.g.~${+c}$ from $\UU$, obtaining $\UU'=\{+d\}$, then $\UU'(\db)=\{a,b,d\}$, and:
  \begin{itemize}
  \item $\UU'(\db)\models\body(\ref{rfok1})$, but ${+c}\not\in\head(\ref{rfok1})$, and likewise for $\ref{rfok3}$;
  \item $\UU'(\db)\not\models\body(\ref{rfok2})$ and $\UU'(\db)\not\models\body(\ref{rfok4})$,
  \end{itemize}
  and thus there is no support for ${+c}$ in $\UU$.
  In this case, there is also no support for ${+d}$.
\end{example}

\citet{tplp/CaropreseT11} discovered that there can be founded repairs exhibiting \emph{circularity of support}.
The following example is due to \citet{tase/Cruz-FilipeEGN13}.

\begin{example}
  \label{ex:founded}
  Consider the following set of AICs $\eta$, expressing that $a$ and $b$ are equivalent (and if this is not the case, then they should both become false) and that $c$ should be true whenever $a$ or $b$ is true.
  \begin{align}
    a\land\neg b &\supset{-a} \label{rf5} \\
    \neg a\land b &\supset{-b} \label{rf6} \\
    a\land\neg c &\supset{+c} \label{rf7} \\
    b\land\neg c &\supset{+c} \label{rf8} 
  \end{align}
  Suppose that the database is $\db=\{a,b\}$.
  There are two repairs for $\fulldb$: $\UU_1=\{{-a},{-b}\}$ and $\UU_2=\{{+c}\}$.
  Intuitively, the rules suggest that $\UU_2$ should be the preferred repair, since it includes the action suggested by the only AICs that is not satisfied, and indeed $\UU_2$ is founded (removing its only element yields $\emptyset$, and as we observed both \ref{rf7} and \ref{rf8} provide support for ${+c}$ given the state of $\db$).

  However, $\UU_1$ is also a founded repair.
  If we remove ${-a}$ from $\UU_1$, we obtain $\UU'=\{{-b}\}$, and $\UU'(\db)=\{a\}$, where \ref{rf5} is applicable and ${-a}\in\head(\ref{rf5})$.
  Dually, if we remove ${-b}$ we obtain $\UU''=\{{-a}\}$, and $\UU''(\db)=\{b\}$; now \ref{rf6} is applicable, and ${-b}\in\head(\ref{rf6})$.
  Thus both actions in $\UU_1$ are founded, hence this is a founded repair.
\end{example}

%BART: I removed this, since we will define groundednesss with this argument :p 
The problem in this example is that the property of being a founded repair only excludes \emph{individual} actions that are not supported by the remaining ones, rather than \emph{sets} of actions with this characteristic.
In order to avoid this unwanted characteristic, \citet{tplp/CaropreseT11} proposed considering justified repairs, which we now define.\footnote{\citet{tplp/CaropreseT11} never formally define circularity of support, or discuss why justified repairs avoid it. We give an informal argument to sustain this claim in the discussion after Proposition~\ref{prop:justified} below.}

\begin{definition}[\citenb{tplp/CaropreseT11}]
  Let \UU be a set of update actions and \fulldb a database. 
  \begin{itemize}
   \item The \emph{no-effect actions} with respect to \db and \UU, $\neff_\db(\UU)$, are the actions that change neither \db, nor $\UU(\db)$.
     \begin{align*}\neff_\db(\UU) &= \{\add a\mid a \in \db \cap \UU(\db)\} \cup \{\remove a \mid a \not\in \db \cup \UU(\db)\}\\
       &=\{\alpha \mid \alpha(\db) = \db \land \alpha(\UU(\db))=\UU(\db)\}\,.
     \end{align*}
   \item The set of \emph{non-updatable literals} of an AIC $r$, $\nup(r)$, contains all body literals of $r$ that do not occur in the head of $r$.
     \[\nup(r) = \body(r) \setminus \litof\left(\head(r)^D\right)\,.\]
   \item $\UU$ is closed under \aics if for each $r\in \aics$, $\actof(\nup(r))\subseteq \UU$ implies $\head(r)\cap \UU\neq \emptyset$.
   \item \UU is a \emph{justified action set} if it is a minimal superset of $\neff_\db(\UU)$ closed under \aics.
   \item \UU is a \emph{justified (weak) repair} if it is a (weak) repair and $\UU\cup \neff_{\db}(\UU)$ is a justified action set. 
  \end{itemize}

%  \todo{something else releated to this definition. THe notion of ``closed'' as defined in the original paper and also the grounded fixpoints paper is quite surprising. I am in particular surprised by the fact that this does not take \db into account. Consider for instance the following \aics:
%  \[a\aicrule \remove a\]
%  with an empty database. The only $U$ that is closed here is $\{\remove a\}$. 
%  It makes sense, though. And this insight might make some proofs easier later in the paper (sorry, I'm just using this paper as well to record my thoughts)}
\end{definition}
Although the notion of closed set of actions does not take the database into account, its role in the definition of justified weak repairs is as part of the definition of justified action set -- where all actions that do not change the database are included.
In the normalized case that we considered, all justified weak repairs are minimal with respect to set inclusion, i.e., they are justified repairs.

\begin{example}
  In the setting of Example~\ref{ex:founded}, the sets of non-updateable literals are as follows.
  \begin{align*}
    \nup(\ref{rf5}) &= \{\neg b\} \\
    \nup(\ref{rf7}) &= \{a\} \\
    \nup(\ref{rf6}) &= \{\neg a\} \\
    \nup(\ref{rf8}) &= \{b\}
  \end{align*}
  The founded repair $\UU_1$ is not justified.
  First, observe that $\neff(\UU_1)=\{{-c}\}$ (assuming that $a$, $b$ and $c$ are the only atomic formulas in the language).
  Consider $\UU'=\emptyset\subseteq\UU_1$; then $\UU'\cup\neff(\UU_1)=\{{-c}\}$, and $\actof(\nup(r))\not\subseteq(\UU'\cup\neff(\UU_1))$ for every $r\in\eta$; therefore $\UU'\cup\neff(\UU_1)$ is a subset of $\UU_1\cup\neff(\UU_1)$ containing $\neff(\UU_1)$ that is (trivially) closed under $\eta$.

  In contrast, the repair $\UU_2$ is justified.
  In this case, we have $\neff(\UU_2)=\{{+a},{+b}\}$; the only proper subset of $\UU_2$ is again $\UU'=\emptyset$.
  Then both \ref{rf7} and \ref{rf8} satisfy $\actof(\nup(r))\subseteq(\UU'\cup\neff(\UU_2))=\{{+a},{+b}\}$, and in both cases $\head(r)=\{{+c}\}$.
  Since $\head(r)\cap\{{+a},{+b}\}=\emptyset$, we conclude that $\UU'\cup\neff(\UU_2)$ is not closed under $\eta$.
\end{example}

The relation between founded and justified weak repairs was established by \citet{tplp/CaropreseT11}.
\begin{lemma}
  \label{lem:justified-founded}
  Let $\db$ be a database, $\eta$ be a set of AICs over $\db$ and $\UU$ be a set of update actions over $\db$.
  If $\UU$ is a justified weak repair for $\fulldb$, then $\UU$ is a founded weak repair for $\fulldb$.
\end{lemma}

One of the main issues with the notion of justified repair is that it is too restrictive.
Interestingly, this can already be seen by an example originally given by \citet{tplp/CaropreseT11}, which we reproduce below.

\begin{example}
  \label{ex:justified}
  Consider the following set of AICs $\eta$.
  \begin{align}
  a\land b &\supset{-a} \label{rj9} \\
  a\land\neg b &\supset{-a}  \label{rj10} \\
  \neg a\land b &\supset{-b} \label{rj11}
  \end{align}
  Consider the same database $\db=\{a,b\}$ as before.
  This database is inconsistent (it does not satisfy \ref{rj9}), and the only possible repair is $\UU=\{{-a},{-b}\}$.
  Furthermore, this repair is intuitively compatible with $\eta$, since rule \ref{rj9} requires us to remove $a$ from $\db$, which triggers \ref{rj11} and forces us also to remove $b$.

  The repair $\UU$ is founded, but for a different reason: if we remove ${-a}$ from $\UU$, then \ref{rj10} is applicable, and its head includes $-a$, while removing ${-b}$ from $\UU$ makes \ref{rj11} applicable, and its head includes $-b$.
  \citet{tplp/CaropreseT11} considered this as another instance of circularity of support (see~\cite{tase/Cruz-FilipeEGN13} for a discussion).
  The repair $\UU$ is not justified: $\neff(\UU)=\emptyset$, and taking $\UU'=\emptyset$ we have that $\UU'\cup\neff(\UU)=\emptyset$ is again trivially closed under $\eta$ (every rule has non-updateable literals in its body).
\end{example}

A different attempt to resolve the problems of circularity posed by founded repairs, while avoiding the over-restrictiveness of justified repairs, was the introduction of well-founded repairs by~\citet{tase/Cruz-FilipeEGN13} -- a third kind of repairs, motivated by an operational approach directly inspired by the syntax of AICs.


\begin{definition}[\citenb{tase/Cruz-FilipeEGN13}]
 A (weak) repair \UU for \fulldb is \emph{well-founded} if there exists a sequence of actions $\alpha_1,\dots,\alpha_n$ such that $\UU=\{\alpha_1,\dots,\alpha_n\}$ and, for each $i\in\{1,\dots,n\}$, there is a rule $r_i$ such that $U_{i-1}(\db) \models \body (r_i)$ and $\alpha_i \in \head(r_i)$, where $U_{i-1} = \{\alpha_1,\dots,\alpha_{i-1}\}$.
\end{definition}

\begin{example}
  In the setting of Example~\ref{ex:founded}, the only well-founded repair for $\fulldb$ is $\{{+c}\}$, as \ref{rf7}  is the only rule applicable in $\db$.

  Likewise, the repair $\UU$ in Example~\ref{ex:justified} is well-founded, as it can be constructed by applying first \ref{rj9} (introducing ${-a}$) and afterwards \ref{rj11}.
\end{example}

However, well-founded repairs can also behave unexpectedly.
\begin{example}
  \label{ex:well-founded}
  Let $\eta$ be the set of AICs containing
  \begin{align}
  a\land\neg b\land\neg c &\supset{+c} \label{rw12} \\
  \neg a\land\neg b &\supset{+b} \label{rw13} \\%\qquad &
  \neg a &\supset{+a} \qquad \label{rw14}
  \end{align}
  and consider $\db=\emptyset$.
  There are two well-founded repairs for $\fulldb$: $\UU_1=\{{+b},{+a}\}$, obtained by applying first \ref{rw13} and then \ref{rw14}, and $\UU_2=\{{+a},{+c}\}$, obtained by applying first \ref{rw14} and then \ref{rw12}.
  It is arguable that $\UU_2$ is preferable, as it is not reasonable to apply \ref{rw13} when \ref{rw14} is also applicable, since any action that solves the inconsistency detected by \ref{rw14} also repairs \ref{rw13}, but not conversely.
  However, the well-founded semantics for AICs cannot infer this restriction.
\end{example}
In this example, it is interesting to note that $\UU_2$ is a founded repair (\ref{rw14} supports ${+a}$ also after $c$ has been added to $\db$, and \ref{rw12} supports ${+c}$), but $\UU_1$ is not (the action ${+b}$ is not supported, as \ref{rw13} is not applicable once $a$ as been added to $\db$).

The examples above show that there exist both founded weak repairs that are not well-founded, and well-founded weak repairs that are not founded.
These relations were established by \citet{tase/Cruz-FilipeEGN13}, together with the connection to justified weak repairs.
\begin{lemma}
  \label{lem:justified-wf}
  Let $\db$ be a database, $\eta$ be a set of AICs over $\db$ and $\UU$ be a set of update actions over $\db$.
  If $\UU$ is a justified weak repair for $\fulldb$, then $\UU$ is a well-founded weak repair for $\fulldb$.
\end{lemma}

We are also interested in the \emph{shifting property}.
Originally defined by \citet{tcs/MarekT98} in the context of revision programming, this property was later transferred to active integrity constraints \cite{tplp/CaropreseT11}. 
Intuitively, a semantics for AICs possesses the shifting property if uniformly replacing some literals with their duals
preserves the semantics at hand, i.e., if the semantics treats truth and falsity of elements in the database symmetrically.

\begin{definition}
 Let $S\subseteq\atoms$ be a set of atoms and $l$ a literal. The \emph{shift of $l$} with respect to $S$ is defined as 
 \[\shift_S(l) = \left\{\begin{array}{ll}                                                                                                             l&\text{if $l\not\in S$}\\                                                                                                              l^D&\text{otherwise}                                                                                                                                      \end{array}\right.\]
 The shift function is extended to sets of literals, update actions and AICs in the straightforward manner. 
\end{definition}

\begin{definition}
 We say that a semantics for AICs \emph{has the shifting property} if: for all \fulldb and all $S\subseteq \atoms$, \UU is a repair of \fulldb accepted by the semantics if and only if $\shift_S(\UU)$ is a repair of $\langle\shift_S(\db),\shift_S(\aics)\rangle$ accepted by the semantics.
\end{definition}

If a semantics has the shifting property, then we can reduce any situation to the case $\db=\emptyset$ by taking $S=\db$.  All semantics discussed in this section have the shifting property. 

%%% Local Variables: 
%%% mode: latex
%%% TeX-master: "../AFT-semantics-AIC"
%%% End: 
