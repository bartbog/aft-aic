\luis{todo: add examples from my ICLP paper to this section}

\paragraph{Active Integrity Constraints}
We assume a fixed set \atoms of \emph{atoms}. An \emph{interpretation} or \emph{database} is a subset of \atoms. A \emph{literal} is an atom $a$ or its negation $\lnot a$. 
We say $\lnot a$ \longpaper{is the \emph{dual} literal of $a$ and vice versa }\shortpaper{and $a$ are dual literals }and denote the dual of a literal $l$ by $l^D$. 
The satisfaction relation between databases \db and literals is defined as usual: $\db \models a$ if $a\in\db$ and $\db \models \lnot a$ if $\db\not\models a$. 

An \emph{update action} has the form $\add a$ or $\remove a$ with $a\in \atoms$. We call $\add a$ and $\remove a$ \emph{dual actions} and use $\alpha^D$ to denote the dual action of $\alpha$. 
\longpaper{Intuitively, u}\shortpaper{U}pdate actions represent changes to the database: \longpaper{$\add a$ adds $a$, while $\remove a$ removes $a$. Formally} $\add a$ transforms $\db$ to $\db\cup\{a\}$ and $\remove a $ transforms $\db$ to $\db\setminus\{a\}$. 
A set of update actions $\UU$ is \emph{consistent} if it does not contain an action and its dual. A consistent set of update actions \UU acts on a database \db by executing all its actions simultaneously; we denote the result of this operation by $\UU(\db)$. 

Literals and update actions are related by mappings \litof and \actof, where $\litof(\add a)= a, \litof(\remove a) = \lnot a$ and $\actof$ is the inverse of $\litof$.
%, \actof(a) = \add a$ and $\actof(\lnot a) = \remove a$.  
These mappings naturally extend to sets of literals/actions. 

\begin{definition}
 An \emph{active integrity constraint} (AIC) is a rule $r$ of the form
%  \todo{replace $\mid$ by $\lor$?}
  \begin{equation} l_1\land \dots\land l_n\aicrule \alpha_1\mid\dots \mid \alpha_k \label{eq:aic}\end{equation}
  such that $\litof(\alpha_i^D) \in \{l_1,\dots,l_n\}$ for each $i$.
%  
  We call $l_1\land \dots \land l_n$ the \emph{body} of $r$, denoted $\body(r)$, and $\alpha_1\mid\dots \mid \alpha_k$ the head of $r$, denoted $\head (r)$.
\end{definition}

The informal reading of the above rule is: ``If each of the $l_i$ holds in \db, then \db is inconsistent. An allowed fix is to execute one or more of the $\alpha_i$.''
A set of AICs represents constraints a database should adhere to and, in addition, which atoms can be changed in order to fix it. Intuitively, atoms can only be changed if there is some rule that allows it. 


An AIC is \emph{normal} if $k=1$. The \emph{normalization} of an AIC of the form \eqref{eq:aic} is the set of AICs 
\[\{l_1\land \dots\land l_n\aicrule \alpha_i\mid 1\leq i\leq k\}.\]
It follows from the informal explanation above that we expect normalization to preserve semantics. Indeed, this is the case for most semantics of AICs (the notorious exception being the semantics of justified repairs \cite{tplp/CaropreseT11}, which poses several other problems~\cite{tase/Cruz-FilipeEGN13}).
In the current paper, we assume \emph{all AICs are normal}. Extensions of the semantics we define for non-normal AICs can then be obtained through normalization, if needed.

\begin{definition}
 A set of update actions \UU is a \emph{weak repair} for \db and a set \aics of AICs (shortly, for $\fulldb$) if:
 \begin{compactitem}
  \item every action in \UU changes \db, and 
  \item $\UU(\db)\not\models \body(r)$ for each $r\in\aics$.
 \end{compactitem}
A $\subseteq$-minimal weak repair is called a \emph{repair}.
\end{definition}

(Weak) repairs do not take the head of AICs into account, and thus allow arbitrary changes to the database. We now review \longpaper{several other }semantics for AICs that have been defined with the intention to allow only changes explicitly allowed by one of the AICs: founded (weak) repairs \cite{iclp/CaropreseGSZ06}, justified (weak) repairs \cite{tplp/CaropreseT11}, well-founded (weak) repairs \cite{tase/Cruz-FilipeEGN13}, and grounded repairs \cite{iclp/Cruz-Filipe16}.

\begin{definition}[\cite{iclp/CaropreseGSZ06}]
 A set of update actions \UU is \emph{founded} with respect to $\fulldb$ if, for each $\alpha \in \UU$, there is a rule  $r\in \aics$ with $\alpha\in\head(r)$ and such that $\UU'(\db)\models \body(r)$, where $\UU'=\UU\setminus\{\alpha\}$. A \emph{founded (weak) repair} is a (weak) repair that is founded.
\end{definition}

\begin{definition}[\cite{tase/Cruz-FilipeEGN13}]
 A (weak) repair \UU for \fulldb is \emph{well-founded} if there exists a sequence of actions $\alpha_1,\dots,\alpha_n$ such that $\UU=\{\alpha_1,\dots,\alpha_n\}$ and, for each $i\in\{1,\dots,n\}$, there is a rule $r_i$ such that $U_{i-1}(\db) \models \body (r_i)$ and $\alpha_i \in \head(r_i)$, where $U_{i-1} = \{\alpha_1,\dots,\alpha_{i-1}\}$.
\end{definition}

\begin{definition}[\cite{tplp/CaropreseT11}]
  Let \UU be a set of update actions and \fulldb a database. 
  \begin{compactitem}
   \item The \emph{no-effect actions} with respect to \db and \UU, \shortpaper{$\neff_\db(\UU)$, }are the actions that change neither \db, nor $\UU(\db)$.
     \begin{align*}\neff_\db(\UU) &= \{\add a\mid a \in \db \cap \UU(\db)\} \cup \{\remove a \mid a \not\in \db \cup \UU(\db)\}\\
       &=\{\alpha \mid \alpha(\db) = \db \land \alpha(\UU(\db))=\UU(\db)\}\,.
     \end{align*}
   \item The set of \emph{non-updatable literals} of an AIC $r$, $\nup(r)$, contains all body literals of $r$ that do not occur in the head of $r$.
     \[\nup(r) = \body(r) \setminus \litof\left(\head(r)^D\right)\,.\]
   \item $\UU$ is closed under \aics if for each $r\in \aics$, $\actof(\nup(r))\subseteq \UU$ implies $\head(r)\cap \UU\neq \emptyset$.
   \item \UU is a \emph{justified action set} if it is a minimal superset of $\neff_\db(\UU)$ closed under \aics.
   \item \UU is a \emph{justified (weak) repair} if it is a (weak) repair and $\UU\cup \neff_{\db}(\UU)$ is a justified action set. 
  \end{compactitem}

%  \todo{something else releated to this definition. THe notion of ``closed'' as defined in the original paper and also the grounded fixpoints paper is quite surprising. I am in particular surprised by the fact that this does not take \db into account. Consider for instance the following \aics:
%  \[a\aicrule \remove a\]
%  with an empty database. The only $U$ that is closed here is $\{\remove a\}$. 
%  It makes sense, though. And this insight might make some proofs easier later in the paper (sorry, I'm just using this paper as well to record my thoughts)}
\end{definition}
Although the notion of closed set of actions does not take the database into account, its role in the definition of justified weak repairs is as part of the definition of justified action set -- where all actions that do not change the database are included.


\citet{tcs/MarekT98} defined in the context of revision programming \emph{the shifting property}; this property was later transferred to active integrity constraints \cite{tplp/CaropreseT11}. 
Intuitively, a semantics for AICs possesses the shifting property if uniformly replacing some literals with their duals
preserves the semantics at hand\longpaper{, i.e., if the semantics treats truth and falsity of elements in the database symmetrically}. 

\begin{definition}
 Let $S\subseteq\atoms$ be a set of atoms and $l$ a literal. The \emph{shift of $l$} with respect to $S$ is defined as 
 \[\shift_S(l) = \left\{\begin{array}{ll}                                                                                                             l&\text{if $l\not\in S$}\\                                                                                                              l^D&\text{otherwise}                                                                                                                                      \end{array}\right.\]
 The shift function is extended to sets of literals, update actions and AICs in the straightforward manner. 
\end{definition}

\begin{definition}
 We say that a semantics for AICs \emph{has the shifting property} if: for all \fulldb and all $S\subseteq \atoms$, \UU is a repair of \fulldb accepted by the semantics if and only if $\shift_S(\UU)$ is a repair of $\langle\shift_S(\db),\shift_S(\aics)\rangle$ accepted by the semantics.
\end{definition}

If a semantics has the shifting property, then we can reduce any situation to the case $\db=\emptyset$ by taking $S=\db$. 
\paragraph{Lattices, Operators and Fixpoints}
A \emph{partially ordered set (poset)} $\langle L,\leq\rangle$ is a set $L$ equipped with a partial order $\leq$, i.e., a reflexive, antisymmetric, transitive relation. 
As usual, we write $x<y$ as abbreviation for $x\leq y \land x\neq y$.
If $S$ is a subset of $L$, then $x$ is an \emph{upper bound}, respectively a \emph{lower bound} of $S$ if for every $s\in S$, it holds that $s\leq x$, respectively $x\leq s$. 
An element $x$ is a \emph{least upper bound} (respectively \emph{greatest lower bound} of $S$) if it is an upper bound that is smaller than every other upper bound (resp.\ a lower bound that is greater than every other lower bound).
If $S$ has a least upper bound (resp.\ a greatest lower bound) we denote it $\lub(S)$ (resp.\ $\glb(S)$).
As is custom, we sometimes call a greatest lower bound a \emph{meet}, and a least upper bound a \emph{join} and use the related notations $\bigand S = \glb(S)$, $x\land y=\glb(\{x,y\})$, $\bigor S = \lub(S)$ and $x\lor y=\lub(\{x,y\})$.
% The relation $\leq$ is a \emph{total order} if for every $x,y\in L$, $x\leq y$ or $y\leq x$. 
% A \emph{chain} is a subset $S$ of $L$ such that $\leq$ is a total order in $S$. We call $\langle L,\leq\rangle$ \emph{chain complete} if each of its chains has a least upper bound and a greatest lower bound. 
%  We call $\langle L,\leq\rangle$ a \emph{bounded lattice}  if every finite subset of $L$ has a least upper bound and a greatest lower bound. 
 We call $\langle L,\leq\rangle$ a \emph{complete lattice}  if every subset of $L$ has a least upper bound and a greatest lower bound. 
%  A complete lattice is always chain complete.
% A chain complete poset has a least element $\bot$. 
A complete lattice has both a least element $\bot$ and a greatest element $\top$. 

Since we apply our results to (finite) databases, for the sake of simplicity we assume $L$ to be \emph{finite} in this text. All presented results easily generalize to the infinite setting as well. 

% A lattice is \emph{distributive} if $\land$ and $\lor$ distribute over each other. A bounded lattice is \emph{complemented} if every element $x\in L$ has a \emph{complement}: an element $\lnot x \in L$ satisfying $x\land \lnot x  = \bot$ and $x\lor \lnot x =\top$.
% A \emph{Boolean lattice} is a distributive complemented lattice. 
% In a Boolean lattice, for every $x,y \in L$, it holds that $x=(x\land y) \lor (x\land \lnot y)$.


An operator $O:L\to L$ is \emph{monotone} if $x\leq y$ implies that $O(x)\leq O(y)$.
%and \emph{anti-monotone} if $x\leq y$ implies that $O(y)\leq O(x)$. 
An element $x\in L$ is 
%a \emph{prefixpoint}, 
a \emph{fixpoint}
%, a \emph{postfixpoint} 
of $O$ 
%if $O(x)\leq x$, respectively 
$O(x)=x$.
%, $x\leq O(x)$. 
Every monotone operator $O$ in a %chain complete poset 
complete lattice has a least fixpoint, denoted $\lfp(O)$, which is 
% also $O$'s least prefixpoint and 
the limit (the least upper bound) of the increasing sequence $(x_i)_{i \in \nat}$ defined by $x_0=\bot$ and $x_{i+1} = O(x_i)$. 
% \begin{compactitem}
% 	\item $x_0=\bot$,
% 	\item $x_{i+1}=O(x_i)$, for successor ordinals $i+1$,
% 	\item $x_\lambda=\lub(\{x_i\mid i<\lambda\})$, for limit ordinals $\lambda$.
% \end{compactitem}


\mycitet{GroundedFixpoints} called a point  $x\in L$ \emph{grounded} for $O$ if, for each $v\in L$ such that $O(v\land x)\leq v$, it holds that $x\leq v$. Later, they generalized this notion to \emph{partial} grounded fixpoints \mycite{PartialGroundedFixpoints}. 
%   They called a point $x\in L$ \emph{strictly grounded} for $O$ if there is no $v\in L$ such that $v<x$ and $O(v)\land x \leq v$.
They explained the intuition underlying these concepts under the assumption that the elements of $L$ are sets of ``facts'' of some kind and the $\leq$ relation is the subset relation between such sets:
in this case, a point $x$ is grounded if it contains only facts that are sanctioned by the operator $O$, 
in the sense that if we remove them from $x$, then the operator will add at least one of them again. 
% They also called a point $x\in L$ \emph{
% They showed that for Boolean lattices, the notions of \emph{groundedness} and \emph{strict groundedness} coincide. 
% In this text, we are only concerned with Boolean lattices and hence freely switch between the two definitions. 

\luis{do we want to talk about strictly grounded and grounded and that they coincide in our context? I'm not sure I don't use both characterizations in the ICLP stuff -- will still check}


\paragraph{Approximation Fixpoint Theory}


Given a lattice $L$, approximation fixpoint theory (AFT) \cite{DeneckerMT00} uses the bilattice 
$L^2$.  We define two \emph{projection} functions for pairs as usual:
$(x,y)_1=x$ and $(x,y)_2=y$.  Pairs $(x,y)\in L^2$ are used to
approximate elements in the interval $[x,y] = \{z\mid x\leq
z\wedge z\leq y\}$. We call $(x,y)\in L^2$ \emph{consistent} if $x\leq 
y$, that is, if $[x,y]$ is non-empty, and use $L^c$ to denote the set
of consistent elements. Elements $(x,x) \in L^c$ are called
\emph{exact}; they constitute the embedding of $L$ in $L^2$.  We sometimes abuse notation and use the tuple $(x,y)$
and the interval $[x,y]$ interchangeably.  The \emph{precision
  ordering} on $L^2$ is defined as $(x,y) \leqp (u,v)$ if $x\leq u$
and $v\leq y$. In case $(u,v)$ is consistent, this means that $(x,y)$
approximates all elements approximated by $(u,v)$, or in other words
that $[u,v]\subseteq [x,y]$.  If $L$ is a complete lattice, then
$\langle L^2,\leqp\rangle$ is also a complete lattice.
  
% \nomenclature[leqp]{$\leqp$}{The precision ordering on $L^2$}
% \nomenclature[A]{$A$}{An approximator of $O$}


AFT studies fixpoints of lattice operators $O:L\ra L$ through operators approximating $O$.
 An operator $A: L^2\to L^2$  is an \emph{approximator} of $O$ if it is \leqp-monotone,  and has the property that $A(x,x) = (O(x),O(x))$ for all $x$. %[x',y']$, where $(x',y')=A{x,x}$.
Approximators are internal in $L^c$ (i.e., map $L^c$ into $L^c$).
As usual, we often restrict our attention to \emph{symmetric} approximators: approximators $A$ such that, for all $x$ and $y$, $A(x,y)_1 = A(y,x)_2$.%\cite{lpnmr/DeneckerV07}.
\citet{DeneckerMT04} showed that the consistent fixpoints of interest (supported, stable, well-founded) are uniquely determined by an approximator's restriction to $L^c$, hence, sometimes we only define approximators on $L^c$. 

AFT studies fixpoints of $O$ using fixpoints of $A$. 
 \begin{compactitem}
  \item The \emph{$A$-Kripke-Kleene fixpoint} is the $\leqp$-least fixpoint of $A$, and it approximates all fixpoints of $O$. 
\item A \emph{partial $A$-stable fixpoint} is a pair  $(x,y)$ such that $x=\lfp(A(\cdot,y)_1)$ and $y=\lfp(A(x,\cdot)_2)$, where $A(\cdot,y)_1$ denotes the operator $L\to L:x\mapsto A(x,y)_1$ and analogously for $A(x,\cdot)_2$. 
\item The \emph{$A$-well-founded fixpoint} is the least precise partial $A$-stable fixpoint. 
\item  An \emph{$A$-stable fixpoint} of $O$ is a fixpoint $x$ of $O$ such that $(x,x)$ is a partial $A$-stable fixpoint. This is equivalent to the condition that $x=\lfp(A(\cdot, x)_1)$.
 \end{compactitem}

The $A$-Kripke-Kleene fixpoint of $O$ can be constructed as the limit of any monotone induction of $A$. 
For the $A$-well-founded fixpoint, a similar constructive characterization has been worked out by \citet{lpnmr/DeneckerV07}:

\begin{definition}\label{002:def:refinement}
An \emph{$A$-refinement} of $(x,y)$ is a pair $(x',y')\in L^2$ satisfying one of the following two conditions:
\begin{enumerate}
	\item $(x,y)\leqp(x',y')\leqp A(x,y)$, or \label{first}
	\item $x'=x$ and  $A(x,y')_2\leq y'\leq y$. \label{second}
\end{enumerate}
An $A$-refinement is \emph{strict} if $(x,y)\neq (x',y')$.
\end{definition}

We call the first type (\ref{first}.) of refinements \emph{application refinements} and the second type (\ref{second}.) \emph{unfoundedness refinements}. If $(x',y')$ is an $A$-refinement of $(x,y)$ and $A$ is clear from the context, we often write $(x,y)\to(x',y')$.
%
% \nomenclature[alp]{$\alpha,\beta$}{Ordinal numbers}
% \nomenclature[lam]{$\lambda$}{A limit ordinal}

 \begin{definition}
 A \emph{well-founded induction} of $A$  is a sequence 
$(x_i,y_i)_{i\leq n}$
with $n\in\nat$ such that 
\begin{compactitem}
	\item $(x_0,y_0) = (\bot,\top)$;
	\item $(x_{i+1},y_{i+1})$ is an A-refinement of $(x_{i},y_{i})$, for  all $i<n$.
% 	\item $(x_{\lambda},y_{\lambda})= \lub_{\leqp} \{(x_i,y_i)\mid i<\lambda\}$
% 	      for each limit ordinal $\lambda\leq\beta$.
\end{compactitem}
A well-founded induction is \emph{terminal} if its limit $(x_n,y_n)$ has no strict $A$-refinements.
\end{definition}
A well-founded induction is an algebraical generalization of the well-founded model construction defined by \citet{GelderRS91}. 
The first type of refinement corresponds to making a partial structure more precise by applying Fitting's immediate consequence operator; the second type of refinement corresponds to making a structure more precise by eliminating an unfounded set. 
For a given approximator $A$, there are many different terminal well-founded inductions of $A$.
\citet{lpnmr/DeneckerV07}  showed that they all have the same limit, which equals the $A$-well-founded fixpoint of $O$. Furthermore, if $A$ is symmetric, then the $A$-well-founded fixpoint of $O$ (in fact, every tuple in a well-founded induction of $A$) is consistent.
% Well-founded inductions that only use the first sort of refinement converge to the $A$-Kripke-Kleene fixpoint. 

\todo{Include section on ultimate approximators?}
% The precision order can be pointwise extended to the family of approximators of $O$. It then follows that more precise approximators have a more precise well-founded fixpoint and that they have more stable fixpoints. 
% \cite{DeneckerMT04} showed that there exists a most precise approximator, $U_O$, called the ultimate approximator of $O$. 
% This operator is defined by \[U_O: L^c\to L^c: (x,y)\mapsto (\bigand O([x,y]), \bigor O([x,y])).\]
% % \nomenclature[UO]{$U_O$}{The ultimate approximator of $O$}
% Here, we used the notation $O(X) = \{O(x)\mid x\in X\}$ for a set $X\subseteq L$.
%  It then follows that for every
% approximator $A$, all  $A$-stable fixpoints are $U_O$-stable fixpoints, and  the $U_O$-well-founded fixpoint is always more precise than the $A$-well-founded fixpoint.  
% We refer to $U_O$-stable fixpoints as \emph{ultimate stable fixpoints} of $O$ and to the $U_O$-well-founded fixpoint as the \emph{ultimate well-founded fixpoint} of $O$.
% Semantics defined using the ultimate approximator have as advantage that they only depend on $O$ since the approximator can be derived from $O$.
% % In this paper, we will focus only on ultimate approximations. More specifically, we will show that for auto-epistemic logic, 
% % whose semantics is defined in terms of Approximation Fixpoint Theory, the ultimate approximator is not precise enough!
% % We will develop an alternative fixpoint theory that is more precise than the current one, and show that with this theory, we obtain several desirable properties in logics with a fixpoint semantics.

When we introduce semantics for AIC based on AFT in the next section, we provide examples of the different constructions considered here. 

%%% Local Variables:
%%% mode: latex
%%% TeX-master: "../AFT-semantics-AIC.tex"
%%% End:
