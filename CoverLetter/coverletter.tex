\documentclass{article}
\usepackage{natbib}
\usepackage{url}
\title{Cover Letter --- Fixpoint Semantics for Active Integrity Constraints}

\begin{document}
\maketitle
Thanks for the helpful reviews. We attach the revised version of our paper. 
 We fixed all spelling mistakes, and notational issues found by the reviewers.
Below we explain the changes we made, and we give our response to the criticisms raised by the reviewers. 


% \section{In response to the criticisms}
	\begin{enumerate}
	 \item Reviewer 2 states
	 \begin{quote}
	  ``In the current paper, we assume that, unless explicitly mentioned otherwise, all AICs are normal.
Extensions of the semantics we define for non-normal AICs can be obtained through normalization, if needed.''
Well, I think that the correct way to proceed is to define the semantics for general (non-normal) AICs and
then to prove that a set of general AICs is equivalent to its normalized version wrt that semantics.
	 \end{quote}
         We understand the referee's point. There is a major reason for not doing this, namely that several of the reviewers already raised the point that notation is sometimes heavy. Restricting to the ``normal'' case is one way to avoid even more notational blow-ups. If the only purpose of defining a semantics for a slightly larger language is showing that it coincides with our semantics for normal AICs, we do not find it worth the trade-off. To us, non-normal AICs are nothing more than syntactic sugar, and hence, are not a topic of semantical consideration. 


	 \item There was a general consensus among the reviewers that more examples are needed throughout the paper. We have taken this issue into account and added several examples. More specifically:
	  \begin{itemize}
	  \item We added a real-world example in the motivation in the introduction.
	  \item We moved preliminaries on logic programming to Section 4. This way, Section 4 now contains examples of the various fixpoints defined in AFT, as requested by Reviewer 2. 
	  \item We added some minor examples in Sections 5-6 to clarify points raised by reviewers. 
	  \end{itemize}
	  
	  \item \label{it:abuse}Several of the reviewers had an issue with the notational abuse where we identify a set of update actions satisfying certain conditions with a set of atoms. We have now removed this identification completely and always work with sets of update actions.  This resulted in changes throughout sections 5-8.
	  \item \label{it:notation} Some reviewers complained about notation being too heavy. This has also been resolved, or at least improved. First of all, since the identification (from the previous point) is no longer made, notation is at least more uniform. Secondly, auxiliary (notation heavy) concepts such as $\mathit{supp}_{\mathcal{U},\mathit{DB},\eta}^{ch}(a)$ $\mathit{supp}_{\mathcal{U},\mathit{DB},\eta}^{keep}(a)$ are now no longer needed due to directly working with actions instead of atoms. In general, notation is more streamlined and less heavy. 
	  \item Reviewer 2 seemed to think a couple of our examples/... are wrong. However, these remarks were all due to confusion that arose because of the heavy notation. 
	  We have clarified this in the revised version (cfr also points \ref{it:abuse} and \ref{it:notation}). Some details: 
	  \begin{itemize}
	  \item \begin{quote}I think that Example 6.3 is wrong. After applying U, the fact a is true in DB.
Therefore, ch a = -a.... \end{quote} and 
\begin{quote}
 Definition 6.2 is not clear. After updating a database DB with U (which is a partial action set) the
fact 'a' could be unknown, that is it's not clear whether it belongs to DB or not.
Well, in this case 'ch a' is not defined!
\end{quote}
Both of these comments are explained by the fact that the definition of ch a only depends on the original DB, not on $U(DB)$, hence, if $a\not\in DB$, then  $ch\ a$ remains equal to $+a$, it never becomes $-a$ or unknown/undefined. 
	  \item \begin{quote}
	         Definition 5.2 looks formally wrong. If Z is the set of all possible sets of update actions, it should be a function from Z to Z (and not from $2^{At}$ to $2^{At}$).
	        \end{quote}
	         The confusion here is due to the fact that we identified elements of $Z$  with subsets of $At$ and used them interchangeably. This is no longer the case.
	      
\end{itemize}
\item There were some general comments regarding the intuitions behind Definition 5.2 and the $\uplus$ operator. We have clarified this, by means of defining three desired properties (inertia, cancellation and completeness) of our operator and then showing that these three properties characterize our operator completely. 
\item There were some more suggestions for related work, namely regarding a comparison between our translation from AICs to logic programs in Section 7 and a similar translation defined by Caroprese et al. We have added such a comparison at the end of Section 7.
\item The first reviewer requested to add a discussion on limitations of our translations. We did this at the end of Section 7. 
	  \end{enumerate}
% 
% \section{Small changes}
%  The only suggestions we did not incorporate are the following: 
% \begin{itemize}
% 	\item 
% \end{itemize}

\bibliographystyle{../idp-latex/stylesheets/elsevier/elsarticle-harv}
\bibliography{../idp-latex/krrlib}

\end{document}



